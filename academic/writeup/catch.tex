\documentclass[preprint]{sigplanconf}

\usepackage{amsmath}
\usepackage{alltt}

\newcommand{\T}[1]{\texttt{#1}}
\newcommand{\tup}[1]{\langle #1 \rangle}

\begin{document}

\conferenceinfo{ICFP '07}{date, City.} %
\copyrightyear{2007} %
\copyrightdata{[to be supplied]}

\titlebanner{banner above paper title}        % These are ignored unless
\preprintfooter{short description of paper}   % 'preprint' option specified.

\title{Catch}
\subtitle{A Technical Overview}

\authorinfo{Neil Mitchell}
           {York}
           {ndm}
\authorinfo{Colin Runciman}
           {York}
           {colin}

\maketitle

\begin{abstract}
A Haskell program may fail at runtime with a pattern-match error if the program
has any incomplete (non-exhaustive) patterns in definitions or case
alternatives. This paper describes a static checker that allows non-exhaustive
patterns to exist, yet ensures that a pattern-match error does not occur. It
describes a constraint language that can be used to reason about pattern
matches, along with mechanisms to propagate these constraints between program
components.
\end{abstract}

\category{CR-number}{subcategory}{third-level}

\terms
term1, term2

\keywords
keyword1, keyword2

\section{Introduction}

Introduction, todo.

\subsection{Road map}

The Catch tool can be seen as 3 entirely separate sections. Initially a program
is translated into the reduced Haskell language \S\ref{chap:yhc}. Second, the
program is transformed into a simpler first-order program with the same
semantics \S\ref{chap:defunc}. Third, a constraint language
\S\ref{chap:constraints} is used to analyse the program \S\ref{chap:backward}.
Some results are presented \S\ref{chap:results} along with some concluding
remarks \S\ref{chap:conc}.

\section{Haskell: Sugarless and Typeless}
\label{chap:yhc}

The full Haskell language is a bit unwieldy for analysis. In particular the
syntactic sugar complicates analysis by introducing more types of expression to
consider. The checker works instead on a simplified language, a core to which
other Haskell programs can be reduced. This core language is a functional
language, making use of case expressions, function applications and algebraic
data types.

\subsection{Yhc Core}

In order to generate a simplified language, it is natural to start with a full
Haskell compiler, and we chose Yhc, a fork of nhc. The internal language of Yhc
is called PosLambda -- a simple variant of lambda calculus without types, but
with positional information. Since Yhc is a relatively simple Haskell compiler,
relying on simple desugaring transformations, this means that the generated
PosLambda is reasonably close to the original Haskell in its structure.

PosLambda does have some features that have no natural parallel in Haskell, for
example there is a FatBar construct, that is used for compiling pattern
matches. The PosLambda language is also very much internal to the compiler. We
have introduced a new Core language to Yhc, very similar to both PosLambda, and
a simple subset of Haskell. This Core language is not explicitly typed, and has
very few constructs in it. We have also written a library, Yhc.Core which is
used by Yhc to generate these Core files, and can also be used by external
programs to read and manipulate the generated Core.

Another important simplification is that all case's are now complete, and where
an error can possibly occur an explicit call to error is inserted, with the
string representing the source position of the error.

\subsection{The Dictionary Transformation}

Most of the desugaring transformations performed by Yhc are at a very local
level -- for example \T{f = (+1)} becomes \T{f v1 = flip ((+) v1) (fromInteger
1)}. The biggest exception to this is the typeclass transformation, which is
based on a method called dictionary passing.

For example, take the following code:

\begin{alltt}
 f :: Eq a => a -> a -> Bool
 f x y = x == y || x /= y
\end{alltt}

This is translated by Yhc into:

\begin{alltt}
 f :: (a -> a -> Bool, a -> a -> Bool) -> a -> a -> Bool
 f dict x y = (||) (((==) dict) x y) (((/=) dict) x y)

 (==) (a,b) = a
 (/=) (a,b) = b
\end{alltt}

Essentially, a tuple is passed around containing the methods in dictionary.


\subsection{Case and Paths}

The Catch internal language is differs in its representation of Case, and
introduces Paths. First we introduce a motivation for the path construct, its
full use will only be shown later in the constraint section. Let us take the
example of head, in the Core language of Yhc this comes out approximately as:

\begin{alltt}
head x = case x of
            (a:as) -> a
            [] -> error "here"
\end{alltt}

Now, let us consider the case where we know that \T{x} is a \T{(:)} constructor
before executing \T{head}. Now, we can rewrite this code as:

\begin{alltt}
 headNonEmpty x = case x of
                     (a:as) -> a
\end{alltt}

Unfortunately we are now using the case expression for two different purposes,
to select the elements in a known data structure, and to test the type of a
data structure at runtime. For the purposes of analysis this is even worse, and
complicates things considerably.

What if we were to introduce an extra construct, such as \T{x.hd} meaning
\T{x}, taking the \T{hd} component of a \T{(:)}. Now we can rewrite \T{head} of
a known \T{(:)} constructor as:

\begin{alltt}
 headNonEmpty x = x.hd
\end{alltt}

It is impossible to have a \T{x.hd} where \T{x} is \T{[]}, this is not a
runtime error, but a static guarantee imposed on the program. With this new
notation, we can now rewrite the standard head as:

\begin{alltt}
 head x = case x of
            (:) -> x.hd
            [] -> error "here!"
\end{alltt}

Note that now we have introduced paths, there is no need to specify names for
each of the constructors given in a case statement. There is also no need to
introduce fresh variables within a function, only the variables can be
manipulated.

\section{First Order Haskell}
\label{chap:defunc}

Having a simple Haskell-like language, there are essentially 2 features that
Haskell has that can be lived without, laziness and higher-order functions. It
is possible to convert Haskell to be a first order language by using
defunctionalization, and it is possible to convert Haskell to be a lazy
language by continuation passing. For the purposes of this checker, laziness is
a nice property, as it allows the code to be treated much like a set of maths
equations, without worrying about evaluation order etc. Higher order functions
are less helpful, and in fact obscure the flow of control within a program --
their removal is beneficial.

\subsection{Reynold's style defunctionalization}

Reynold's style defunctionalization is a simple method of generating a first
order program from a higher order one. Taking the following example:

\begin{alltt}
 map f x = case x of
                [] -> []
                (a:as) -> f a : map f as
\end{alltt}

Here f is a higher order input, of type \T{(a -> b)}. Defunctionalisation works
by creating a data type to represent all possible f's, and using that. For
example:

\begin{alltt}
 data Functions = Head | Tail

 apply Head x = head x
 apply Tail x = tail x

 map f x = case x of
     [] -> []
     (a:as) -> apply f a : map f as
\end{alltt}

Now all calls to map head would be replaced with map Head, and Head is a first
order value. This scheme can easily be extended to currying, let us take the
\T{add} function, which adds two \T{Int}'s:

\begin{alltt}
 data Functions = Head | Tail | Add0 | Add1 Int

 apply Add0 n = Add1 n
 apply (Add1 a) b = add a b
\end{alltt}

There are a couple of things to note about this approach. One is that while
this is still type safe, to do type checking would require a dependently typed
language. This is unfortunate, but for our checker (which does not use type
information), this is acceptable. The unacceptable aspect is the creation of an
apply function, whose semantics are very general. This essentially introduces a
bottleneck through which various properties must be proven. Asking questions,
such as is the result of apply a Cons or a Nil, are confusing.

As such, while Reynold's style defunctionalization is acceptable, it is not the
ideal method for removing higher order functions.

\subsection{Specialisation}

Often when a function is called, there is some information known about the
arguments -- for example one may be a constant. The way that the GHC program
makes use of this information is by inlining heavily, unfortunately sometimes
there is a good amount of information available, but the function is unsuitable
for inlining. Specialisation solves this problem neatly.

Let us consider for a minute the function \T{map}, defined as above. Now let us
consider the application \T{map f []}. Here it is not a great idea to inline
map, and GHC is unable to simplify this definition\footnote{GHC is able to
simplify the \T{map} provided in the standard Prelude using rewrite rules,
however by defining \T{map'} with the above definition, no simplification is
made}.

However, using specialisation, it is easy to generate \T{map} specialised with
\T{[]} as:

\begin{alltt}
 map f = []
\end{alltt}

Of course, now even a very conservative inline pass will succeed with this,
resulting in \T{[]} as the end result.

The notation used for specialisation is given as follows. The original version
of map takes two arguments, and is given as \T{map<?,?>} -- i.e. \T{map} with
two arguments, and no additional information. The version specialised on \T{[]}
is given as \T{map<?,[]>}. Note that here there is no second argument, the
specialised version wraps this up.

Now let us consider \T{map f (x:xs)}, this specialised in the same manner to
use the function:

\begin{alltt}
 map<?,?:?> f x xs = f x : map<?,?> xs
\end{alltt}

Note that the recursive case calls the standard map, as it cannot deduce any
information about xs. Obviously this has the potential for non-termination, say
the function \T{map<?,?:(?:(?: $\ldots$} was required. The way to make this
system terminating is to demand that the type must be new, for example it is
impossible to specialise the tail of a \T{(:)}, since the type of the tail is
the same as the initial list. This simple mechanism ensures termination, and at
the same time promotes good simplification of complex argument structures.

However, coming back to the original point, removing higher order functions.
These can be specialised in exactly the same way, for example \T{map head x}
can use the specialised version \T{map<head,?>}. Curried functions can be used
in the specialisation as well, \T{map<add ?,?>} is an example.

One particular usage of this specialisation treatment is the removal of the
dictionary transformation. For example, a function requiring Int's might be
specialised to \T{f<(intEq,intNeq),?,?>}. GHC and hbc both support a
specialisation annotation which achieves this effect, however full
specialisation performs this automatically.

There are of course disadvantages to specialisation, this is essentially a
whole program analysis transformation, and the performance is not stunning.
However, in practice it seems that the generated code corresponds much more
closely to what the original author had meant, and is significantly shorter --
which is not a massive surprise, Jones has shown this result in the specific
context of type classes before.

\section{Detecting Match Errors}

Detecting pattern match errors is equivalent to doing reachability analysis for
the \T{error} function, since Yhc transforms potential pattern match errors to
calls to \T{error}. It is usually very easy at a local level to determine
whether a pattern match will occur, for example with \T{head}, the precondition
for the function to be safe is that the input argument must be a \T{(:)}. What
is hard to determine is whether the input to head is a \T{(:)} or not. Backward
analysis is the process by which this is determined.

Backward analysis works by taking a condition on an expression, and
transforming it to a condition on the free variables in that expression.
Formally this is denoted by:

\[ P'(\text{fv}(x)) \Rightarrow P(x) \]

The process of backward analysis takes $x$ and $P$, and generates $P'$, where
$\text{fv}(x)$ denotes the free variables in $x$. The implication is used
rather than equals because the generated condition has to be sufficient to
ensure the original condition, but may be more strict.

So now let us consider the expression, \T{head (f x)}. The backwards analysis
will convert the condition on \T{f x}, namely that it must be a \T{(:)}, to a
condition on \T{x}. For example, if \T{f} is equivalent to \T{id}, the
identity, then the condition on \T{x} would be that it is a \T{(:)}. However,
if \T{f} was \T{(:[])} -- the function that puts an element into a single
element list -- then the precondition would be $True$, i.e. always safe.

Using this scheme it is necessary to have some formalism for $P$, some
constraint language. The backwards analysis function must also be defined. In
the following two sections, such a language is presented, along with a
backwards analysis function.


\section{A constraint language}
\label{chap:constraints}

Semantics

\subsection{An atomic constraint}

A constraint consists of an expression, a regular expression representing a
path, and a set of constraints.

Introduce regular expressions, quantification


\subsection{Predicates on constraints}

The expressions in constraints have five separate types, case statements,
constructor applications, function applications, path expressions and
variables. The variables are not actually free, but are quantified over the
function in which they occur, who they are arguments to. As a result a complete
constraint is expressed as $\forall f \bullet P(x)$, where $f$ is the function
which provides scope for the free variables in $x$, and $P$ is a proposition
with literal terms described as above.

\subsection{Restricted Regular Expressions}

Restricted according to type.

\section{Backward Analysis Function}
\label{chap:backward}

Given the semantics for Haskell, along with the semantics for the constraint
system it is relatively easy to derive the backward analysis function on all
the syntactic elements, excluding function application. The results for these
are now presented, along with commentary:

Diagram as before

Descriptions as before

After these rules have been applied, the only remaining constraints will be
either on free variables, or function application. Function application is
dealt with by templating, to reduce the problem to one on only free variables.
Once a constraint is on free variables, it is propagated to all the callers of
the given function, and the process is repeated.

\subsection{Templating}

When a constraint of the form $\tup{f \overrightarrow{e},r,c}$ is found, there
are two choices which at first glance can be used. One method is to use the
$\beta$ substitution rule from $\lambda$-calculus, and replace $f$ with its
body. Another alternative is to create fresh variables for each element in
$\overrightarrow{e}$, solve this new problem, and then instantiate the results.

The original Catch program as described in TFP chose the former method, the
program described in this paper chooses the later. The advantage of the first
method is that it very naturally models the $\lambda$-calculus, and therefore
has an intuative feel. The disadvantage is that to acheive a fixed point on the
template instantiation requires some kind fixed point on expressions. By moving
to the second method the algorithm becomes substaintially simpler. It also has
the advantage that the free variables can be chosen in a way such that it is
likely to generate the same instantiation again, and can be satisfied out of a
cache in future invocations. The addition of a cache turns out to be a
substantial time benefit, often changing the underlying complexity of the
algorithm.

The templating method is as follows:

Given a constraint $\tup{f \overrightarrow{e},r,c}$, the constraint %
$\tup{f v_0\ldots v_{\sharp \overrightarrow{e},r,c}}$ is generated and solved
with simple $\beta$ expansion. This method continues as constraints are
generated, but with the knowledge that this identical constraint gives $True$
as its generated condition within this subcomputation. After having solved the
constraint, the result is compared with $True$. If the result matches, it is
used, if it does not then a new computation is attempted with the assumption
that the result matches the first result. This process continues until a fixed
point is found, which is returned as the result. By applying $\wedge$ at each
stage to the previous answer, it is easy to guarantee that the fixed point
found is the greatest fixed point.

The implementation method for this is to use a stack of variables, initially
when a call is encountered the value $True$ and the name of the function and
its constraint are pushed onto the stack. During the computation if a function
which is not recognised is reached, it too is pushed on the stack. However, if
an existing function is spotted, then the value already on the stack is used.
After a computation is finished, is the value matches that which was already on
the stack then a fixed point is found, otherwise the process starts again with
a new initial value.

If a function does not call any functions which are already on the stack, then
its result is not effected by the current stack, and therefore the end result
can be cached. This is of particular importance as it allows many sections of
the computation to be solved once, even though the fixed point may traverse
over that function many times.

\subsection{Propagation}

Once a constraint has been reduced to a proposition of constraints only on free
variables, it is possible to propagate these constraints to the caller of the
function whose free variables are being examined. A fixed point to this process
can be achieved by using $\wedge$ on each iteration, and stopping when no
functions change.

A fixed point can be obtained more quickly by first propagating the functions
at the leaf of the call tree. Once these have found a fixed point, it is only
possible for the functions they call to require further computation with this
expression. This is achieved by constructing a call graph, and taking a
flattening to get an effective order for propagation. This can substantially
reduce the amount of work required.

\section{Implementation concerns}

\subsection{Function guards vs case statements}

While the code operates on case statements, you can flatten them and achieve
better results.

MCase's for easier code

\subsection{Binary Decision Diagrams}

BDD's - why they aren't right, negation in the BDD context.

Advantage of throwing away unused terms quickly.

\section{Results}
\label{chap:results}

\subsection{Small Examples}

Map head, head reverse etc.


\subsection{Case Studies}

adjoxo, soda, clausify

\subsection{Nofib}

The whole nofib benchmark, in detail

\section{Conclusion}
\label{chap:conc}






\appendix
\section{Appendix Title}

This is the text of the appendix, if you need one.

\acks

Acknowledgments, if needed.

\begin{thebibliography}{}

\bibitem{smith02}
Smith, P. Q. reference text

\end{thebibliography}

\end{document}
1-59593-090-6/05/0007
