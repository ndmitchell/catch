\documentclass[preprint]{sigplanconf}

\usepackage{amsmath}

\begin{document}

\conferenceinfo{ICFP '07}{date, City.} %
\copyrightyear{2007} %
\copyrightdata{[to be supplied]}

\titlebanner{banner above paper title}        % These are ignored unless
\preprintfooter{short description of paper}   % 'preprint' option specified.

\title{Catch}
\subtitle{A Technical Overview}

\authorinfo{Neil Mitchell}
           {York}
           {ndm}
\authorinfo{Colin Runciman}
           {York}
           {colin}

\maketitle

\begin{abstract}
A Haskell program may fail at runtime with a pattern-match error if the program
has any incomplete (non-exhaustive) patterns in definitions or case
alternatives. This paper describes a static checker that allows non-exhaustive
patterns to exist, yet ensures that a pattern-match error does not occur. It
describes a constraint language that can be used to reason about pattern
matches, along with mechanisms to propagate these constraints between program
components.
\end{abstract}

\category{CR-number}{subcategory}{third-level}

\terms
term1, term2

\keywords
keyword1, keyword2

\section{Introduction}

Introduction, todo.

\subsection{Road map}

The Catch tool can be seen as 3 entirely separate sections. Initially a program
is translated into the reduced Haskell language \S\ref{chap:yhc}. Second, the
program is transformed into a simpler first-order program with the same
semantics \S\ref{chap:defunc}. Third, a constraint language
\S\ref{chap:constraints} is used to analyse the program \S\ref{chap:backward}.
Some results are presented \S\ref{chap:results} along with some concluding
remarks \S\ref{chap:conc}.

\section{Haskell: Sugarless and Typeless}
\label{chap:yhc}

\subsection{Yhc Core}

The full Haskell language is a bit unwieldy for analysis. In particular the
syntactic sugar complicates analysis by introducing more types of expression to
consider. The checker works instead on a simplified language, a core to which
other Haskell programs can be reduced. This core language is a functional
language, making use of case expressions, function applications and algebraic
data types.

In order to generate a simplified language, it is natural to start with a full
Haskell compiler, and we chose Yhc, a fork of nhc. The internal language of Yhc
is called PosLambda -- a simple variant of lambda calculus without types, but
with positional information. Since Yhc is a relatively simple Haskell compiler,
relying on simple desugaring transformations, this means that the generated
PosLambda is reasonably close to the original Haskell in its structure.

PosLambda does have some features that have no natural parallel in Haskell, for
example there is a FatBar construct, that is used for compiling pattern
matches. The PosLambda language is also very much internal to the compiler. We
have introduced a new Core language to Yhc, very similar to both PosLambda, and
a simple subset of Haskell. This Core language is not explicitly typed, and has
very few constructs in it. We have also written a library, Yhc.Core which is
used by Yhc to generate these Core files, and can also be used by external
programs to read and manipulate the generated Core.

Another important simplification is that all case's are now complete, and where
an error can possibly occur an explicit call to error is inserted, with the
string representing the source position of the error.

\subsection{The Dictionary Transformation}

Most of the desugaring transformations performed by Yhc are at a very local
level -- for example \T{f = (+1)} becomes \T{f v1 = flip ((+) v1) (fromInteger
1)}. The biggest exception to this is the typeclass transformation, which is
based on a method called dictionary passing.

For example, take the following code:

\T{f :: Eq a => a -> a -> Bool; f x y = x == y || x /= y}

This is translated to:

f dict x y = (||) (((==) dict) x y) (((/=) dict) x y)

where

(==) (a,b) = a

and

(/=) (a,b) = b

Essentially, a tuple is passed around containing the methods in dictionary.


\subsection{Case and Paths}

The Catch internal language is differs in its representation of Case, and
introduces Paths. First we introduce a motivation for the path construct, its
full use will only be shown later in the constraint section. Let us take the
example of head, in the Core language of Yhc this comes out approximately as:

head x = case x of
            (a:as) -> a
            [] -> error "here"

Now, let us consider the case where we know that x is a (:) constructor before
executing head. Now, we can rewrite this code as:

head x = case x of
            (a:as) -> a

Unfortunately we are now using the case expression for two different purposes,
to select the elements in a known data structure, and to test the type of a
data structure at runtime. For the purposes of analysis this is even worse, and
complicates things considerably.

What if we were to introduce an extra construct, such as x.hd meaning x, taking
the hd component of a (:). Now we can rewrite head of a known (:) constructor
as:

head x = x.hd

It is impossible to have a x.hd where x is [], this is not a runtime error, but
a static guarantee imposed on the program. With this new notation, we can now
rewrite the standard head as:

head x = case x of
            (:) -> x.hd
            [] -> error "here!"

Note that now we have introduced paths, there is no need to specify names for
each of the constructors given in a case statement. There is also no need to
introduce fresh variables within a function, only the variables can be
manipulated.

\section{First Order Haskell}
\label{chap:defunc}

Specialisation, defunctionalisation, removal of higher order code. Contrast
with inlining. The dictionary transformation, and its inverse.

\section{A constraint language}
\label{chap:constraints}

Semantics

\subsection{An atomic constraint}

Introduce regular expressions, quantification


\subsection{Predicates on constraints}

Include the $\forall$ thing here.

\subsection{Representation of Predicates}

BDD's - why they aren't right, negation in the BDD context.

\section{Backward Analysis}
\label{chap:backward}

Backward analysis

The basic rule schema

Propagation

Templating

\section{Results}
\label{chap:results}

\subsection{Small Examples}

Map head, head reverse etc.


\subsection{Case Studies}

adjoxo, soda, clausify

\subsection{Nofib}

The whole nofib benchmark, in detail

\section{Conclusion}
\label{chap:conc}






\appendix
\section{Appendix Title}

This is the text of the appendix, if you need one.

\acks

Acknowledgments, if needed.

\begin{thebibliography}{}

\bibitem{smith02}
Smith, P. Q. reference text

\end{thebibliography}

\end{document}
1-59593-090-6/05/0007
