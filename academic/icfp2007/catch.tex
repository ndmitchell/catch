\documentclass[preprint]{sigplanconf}

\usepackage{amsmath}
\usepackage{amssymb}
\usepackage{alltt}
\usepackage{url}
\usepackage{natbib}
\usepackage{datetime}

%include lhs2TeX.fmt

% general stuff
\newcommand{\T}[1]{\texttt{#1}}
\newcommand{\tup}[1]{\ensuremath{\langle #1 \rangle}}
\newcommand{\mtxt}[1]{\textsf{#1}}

% examples
\newcounter{exmp}
\setcounter{exmp}{1}
\newcommand{\yesexample}{\subsubsection*{Example \arabic{exmp}}\addtocounter{exmp}{1}}
\newcommand{\noexample}{\hfill$\Box$}

\newcommand{\todo}[1]{\textbf{\textsc{Todo:} #1}}

% code blocks
\newenvironment{code}{\begin{alltt}\small}{\end{alltt}}
\newenvironment{codepage}
    {\begin{minipage}[h]{\textwidth}\begin{code}}
    {\end{code}\end{minipage}}

\newcommand{\K}{\ensuremath{^\ast}} % kleene star
\newcommand{\D}{\ensuremath{\cdot}} % central dot

\renewcommand{\c}[3]{\tup{\T{#1},\T{#2},\T{\{#3\}}}}
\newcommand{\cc}[2]{\c{#1}{$\lambda$}{#2}}

\newcommand{\s}[1]{\ensuremath{_{\tt #1}}} % subscript, in tt font
\newcommand{\g}[1]{\{#1\}} % group, put { } round it
\newcommand{\U}{\textunderscore}
\newcommand{\vecto}[1]{\overrightarrow{#1\;}}
\newcommand{\gap}{\;\;}
\newcommand{\dom}{\text{dom}}


\begin{document}

\conferenceinfo{ICFP '07}{date, City.} %
\copyrightyear{2007} %
\copyrightdata{[to be supplied]}

\titlebanner{\today{} - \currenttime{}}        % These are ignored unless
\preprintfooter{Catch: A Technical Overview}   % 'preprint' option specified.

\title{Catch}
\subtitle{A Technical Overview}

\authorinfo{Neil Mitchell}
           {York}
           {ndm}
\authorinfo{Colin Runciman}
           {York}
           {colin}

\maketitle

\begin{abstract}
A Haskell program may fail at runtime with a pattern-match error if the program has any incomplete (non-exhaustive) patterns in definitions or case alternatives. This paper describes a static checker that allows non-exhaustive patterns to exist, yet ensures that a pattern-match error does not occur. It describes a constraint language that can be used to reason about pattern matches, along with mechanisms to propagate these constraints between program components.
\end{abstract}

% \category{CR-number}{subcategory}{third-level}

% \terms
% term1, term2

% \keywords
% keyword1, keyword2

\section{Introduction}
\label{sec:introduction}

Often it is useful to define pattern matches which are incomplete, for example \T{head} fails on the empty list. Unfortunately programs with incomplete pattern matches may fail at runtime.

Consider the following example:

\begin{code}
risers :: Ord a => [a] -> [[a]]
risers [] = []
risers [x] = [[x]]
risers (x:y:etc) = if x <= y then (x:s):ss else [x]:(s:ss)
    where (s:ss) = risers (y:etc)
\end{code}

A sample execution of this function would be:

\begin{code}
> risers [1,2,3,1,2]
[[1,2,3],[1,2]]
\end{code}

In the last line of the definition, \T{(s:ss)} is matched against the output of \T{risers}. If \T{risers (y:etc)} returns an empty list this would cause a pattern match error. It takes a few moments to check this program manually -- and a few more to be sure one has not made a mistake!

GHC \cite{ghc_manual} 6.6 has a warning flag to detect incomplete patterns, which is named \T{-fwarn-incomplete-patterns}. Adding this flag at compile time reports:

\begin{code}
Warning: Pattern match(es) are non-exhaustive
\end{code}

The Bugs (12.2.1) section of the manual notes that the checks are sometimes wrong, particularly with string patterns or guards, and that this part of the compiler ``needs an overhaul really'' \cite{ghc_manual}.

But the GHC checks are only local. If the function \T{head} is defined, then it raises a warning. No effort is made to check the \textit{callers} of \T{head} -- this is an obligation left to the programmer.

Turning the \T{risers} function over to the checker developed in this paper, the output is:

\begin{code}
> Safe
\end{code}

In addition the checker produces a set of axioms it has proved about Risers, along with an outline of the proof. Fuller details of how the checking is performed follow in Section \ref{sec:walkthrough}.

\subsection{Road map}

This paper first takes the reader on an overview walkthrough of checking the \T{risers} function \S\ref{sec:walkthrough}. Next we introduce a small core functional language, along with a mechanism for reasoning about this language \S\ref{sec:manipulate}, and a constraint language \S\ref{sec:constraint}. Next we discuss how to transform Haskell to this core language \S\ref{sec:transform}.

Having described the underlying system, we move on to an evaluation on various sample programs -- including the \textsf{nofib} benchmark suite \S\ref{sec:results}. We compare our work to other work \S\ref{sec:related} before presenting some concluding remarks \S\ref{sec:conclusion}.

\section{Walkthrough of Risers}
\label{sec:walkthrough}

This section details all the stages that are executed in order to generate a proof that the \T{risers} function in the Introduction does not crash with a pattern match error. No one area is examined in detail, but an overall flavor is given.

\subsection{Modifying Risers for analysis}

There is one restriction on the functions that Catch can analyse -- they must not take any explicitly higher order parameters. The reason for this is to reduce the complexity in the analysis engine. Removing the ability to pass functions as input removes the need to state conditions on functions.

In Haskell \citep{haskell},type classes are typically implemented \citep{type_classes} as dictionaries, meaning that all functions with a type class have a dictionary of higher order functions passed. This means that the function to analyse must not have any outstanding context.

The type of the \T{risers} function is \T{risers :: Ord a => [a] -> [[a]]} -- this can easily be made first-order by making a new function \T{main} the root of the analysis, and providing a concrete type:

\begin{code}
main :: [Int] -> [[Int]]
main x = risers x
\end{code}

\subsection{Conversion to a Core Language}

Rather than analyse full Haskell, the Catch tool analyses a first-order Core language, without lambda's (other than at the top level), partial application or let bindings. While this at first seems like a substantial restriction, a convertor is provided from the full Haskell 98 language to this very restricted language. Full details are provided in section \S\ref{sec:transform}.

After converting to Core Haskell, and renaming the identifiers for ease of human reading, the resulting Core is as shown in Figure \ref{fig:risers_core}. The function \T{risers4} and \T{risers2} correspond to the pattern match in the \T{where}. The function \T{<=} in this example has been specialised to the Int data type, unlike the standard (<=) operator which is a member of the Num class.

\begin{figure}
\begin{code}
risers3 x y = risers4 (risers (x : y))

risers4 x = case x of
    (y:ys) -> (ys, y)
    [] -> error "Pattern Match Failure, 11:12."

risers x = case x of
    [] -> []
    (y:ys) ->  case ys of
         [] -> (y : []) : []
         (z:zs) -> risers2 (risers3 z zs) (y <= z) y

risers2 x y z =  case y of
    True -> (z : snd x) : (fst x)
    False -> (z : []) : (snd x : fst x)
\end{code}
\caption{\T{risers} in the Core language}
\label{fig:risers_core}
\end{figure}

\subsection{Analysis}

Within the Core language every pattern match covers all possible constructors of the appropriate type -- any patterns which were not originally complete have calls to \T{error} inserted. The analysis first starts by finding calls to \T{error}, then trying to prove that these calls will not be reached. In the above example it is easy to see there is only one \T{error} call, which corresponds to a pattern match error which has been desugared.

The analysis determines that if \T{risers4} is called, the argument must always be a \T{(:)}-constructed value for the code to be safe. Following the call chain backwards shows that \T{risers3} calls \T{risers4}, passing \T{(risers (x:y))} as the argument. In this particular case it is possible to see that risers of a \T{(:)} always results in a \T{(:)} - requiring in part the knowledge that \T{risers2} always evaluates to a \T{(:)}. With these axioms it can be shown that the entire program is safe.

Unfortunately the analysis is not blessed with the foresight to know which axioms are required and the direction in which the proof must proceed. The analysis manipulates constraints on expressions until a fixed point is found -- speculatively attempting to discharge constraints as they arise. The exact transformation rules, along with the constraints that can be expressed by the system are the main substance of the analysis.


\section{The Constraints}
\label{sec:manipulate}

This section explains the underlying constraint system in Catch, focusing on how the constraints are put together to express properties of expressions and the data structures they evaluate to. All data structures and equations are presented in Haskell, although should be accessible to all readers.

\subsection{Reduced expression language}

\begin{figure}
\begin{code}
type CtorName = String
type FuncName = String
type PathName = String

data Expr = Var String
          | Path Expr PathName
          | Make CtorName [Expr]
          | Call FuncName [Expr]
          | Case Expr [Alt]

type Alt = (CtorName, Expr)]
\end{code}
\caption{Core Data Type}
\label{fig:core}
\end{figure}

\begin{figure}
\begin{code}
ctors :: CtorName -> [CtorName]
arity :: CtorName -> Int
paths :: CtorName -> [PathName]
pathCtor :: PathName -> CtorName
pathPosn :: PathName -> Int
isRec :: PathName -> Bool
\end{code}
\caption{Miscellaneous functions to manipulate Core}
\end{figure}

The abstract syntax tree for the reduced expression language is given in figure \ref{fig:core}. The Core language chosen for this purpose is like most others, with the exception of paths. The language is first order, has only simple case statements, and only algebraic data types. The evaluation strategy is lazy. All case statements are defined to be complete, with error being introduced where a pattern match error would occur. Every constructor has an arity, which can be obtained with the \T{arity} function. To determine all constructors in a set the \T{ctors} function can be used, for example \T{ctors "True" = ["False", "True"]} and \T{ctors "[]" = ["[]", ":"]}.

For analysis purposes we can ignore the default construction for alternatives - in practice this can always be expressed by a set of alternatives with bound constructors. In practice the default alternative is present in Catch, which results in a performance gain.

\subsubsection{Paths}

The one thing present in our Core language, but not in other languages, is the \textit{path}. Some examples of paths include \T{hd} and \T{tl}. A path asserts that a data value has a given constructor, and then follows its appropriate field. The concrete syntax for paths is \T{x$\multimap$tl}. Some examples of paths:

\begin{code}
(1:2:[])-*hd == 1
(1:2:[])-*tl == (2:[])
[]-*hd -- this is statically impossible
\end{code}

A path has a given constructor, and returns a given field member from that constructor -- this information is available using the functions \T{pathCtor} and \T{pathPosn}.

Constructions such as \T{[]-*hd} are statically disallowed. The original input language to Catch is Haskell, which does not have a path expression, meaning that all path's are introduced by the transformation process. It is the responsibility of the transformations to ensure that invalid paths are not introduced.

To take a simple example, here is the function \T{risers4} from the introductory example, using path's instead of constructors. In general functions are given without the use of paths -- they are an underlying implementation concern which does not change the meaning or understanding of the algorithms.

\begin{code}
risers4 x = case x of
    (:) -> (x-*tl, x-*hd)
    [] -> error "Pattern Match Failure, 11:12."
\end{code}

Our use of paths contrasts with other functional Core language's (for example \citet{ghc_core}). Our choice has a number of merits, in particular they simplify a number of issues. By using path statements, the alternatives in a \T{case} expression do not need to bind variables. This then means that all free variables are declared at the top of a function, and have a scope consisting of the entire function -- since let's expressions are not permitted. The simplification of free variable handling allows a corresponding simplification of the entire analysis. The rules for variables are also simplified -- there are no special cases for variables bound in a case expression.

The name of paths can be automatically generated from a constructor and an index, however if a Haskell program defines a data type with uniquely named fields, then these field names are used. The definition for lists in Catch is:

\begin{code}
data [a] = [] | {hd :: a, tl :: [a]}
\end{code}

\subsubsection{\T{isRec} function}

Detail the purpose of the isRec function, what it does, and why. Also mention that co-recursive types are disallowed.

Invariant, given a finite typed structure, there must be no chain of fields which are not isRec one after another.

\subsubsection{Abstraction}

\begin{figure}
\begin{code}
data Int = Neg | Zero | Pos

any0 = primitive
any2 x y = if any0 then x else y
any3 x y z = any2 x (any2 y z)

x + Zero = x
Zero + x = x
x + y | x == y = x
_ + _ = any3 Neg Zero Pos

x - Zero = x
Zero - Zero = Zero
Zero - Neg = Pos
Zero - Pos = Neg
Neg - Pos = Neg
Pos - Neg = Pos
_ - _ = any3 Neg Zero Pos
\end{code}
\caption{Abstract implementation of integers}
\label{fig:abstract_int}
\end{figure}

The Core language only has algebraic data types, specifically the types missing from a general purpose functional programming language are text characters, integers (bounded and unbounded) and floating point numbers. Catch allows for these programs by abstracting them into algebraic data constructors. Different programs may require different abstractions for the various pattern matches.

Natural numbers are often encoded by Peano numerals, and this idea can easily be extended to integers:

\begin{code}
data Nat = Zero | Succ Nat
data Int = Zero | Minus Nat | Plus Nat
\end{code}

While this abstraction captures all the underlying detail of the number system, it is not detail that can be used during the analysis -- due to the underlying constraint systems discussed in \S\ref{sec:constraints}. A more practically motivated example is given in Figure \ref{fig:abstract_int}, and this is the abstraction that is used in practice. The \T{any0} method is an internally implemented method of Catch, assumed to be any demonic value.

The abstraction of characters is often more practically interesting. The Haskell language standard calls for Unicode character literals -- and depending on the compiler the number of distinct characters varies between 256 and 1114111. There are several possible abstractions of characters:

\begin{code}
data Char = Char
data Char = Alpha | Digit | White | Other
data Char = A .. Z | a .. z | 0 .. 9 | ! .. # | Other
data Char = 00 | 01 | 02 ..
\end{code}

The simplest abstraction is that all characters are the same. A slightly more refined abstraction is to partition the characters based on the character testing functions provided in the \T{Char} module of Haskell. Refining the model further can give special status to characters that commonly occur, giving all uncommon characters the \T{Other} value. Finally, given that \T{Char} is a finite enumeration, the entire range of characters could each be represented distinctly.

In practice, programs often require different levels of abstraction for characters. Many will accept the most basic abstraction, some require certain literals to be represented precisely - for example a noughts and crosses\footnote{Tic-tac-toe, for US readers.} program matches on 'X' and 'O'.

The final issue of abstraction relates to primitive functions, for example the \T{getArgs} function which returns the command line arguments, or the \T{readFile} which reads from the underlying filesystem. In most cases the most appropriate model is to return \T{any0}. In some special cases, for example the CPU tick count, a slightly more precise answer can be given -- namely a positive integer.


\subsection{Constraints}

An expression evaluates to a (potentially infinite) data structure, or to $\bot{}$ caused by either non-termination of pattern match error. If an expression does evaluate to a data structure, then a constraint states the possible forms of data value it may take. A constraint is a set of data structures that a value may match. In order to practically represent a constraint, there are several different concrete representations -- these will be discussed in \S\ref{sec:constraints}.

Given a mechanism of expressing some set of constructors, $E(x) \in C$ states that the expression $x$ must, when evaluated be a member of the set defined by $C$. Since the Core language is typed, the constraint $C$ will only refer to a well typed term. These atomic constraints can be built up into a proposition of constraints, each possibly on different variables.

Since free variables are bound by function calls, it is possible to scope a constraint by making it a proposition. These can then be placed into a proposition. For example, $\forall f, i \in C$ states that for the function $f$, the argument $i$ must be in the set $C$. To take a concrete example, one particular example is $\forall \T{risers4}, x \in \g{\_ : \_}$.

These constraints are sufficient to express many useful properties. Several underlying models are possible, and will be discussed later.

\subsection{Constraints - Pattern Matching}

\begin{figure}
\begin{code}
type Constraint = [Match]

data Match = Match CtorName [Match]
           | Any

data Req = Expr :< Constraint

notin :: CtorName -> Constraint
(|>) :: FieldName -> Constraint -> Constraint
(<|) :: Expr -> Constraint -> Prop Req
\end{code}
\caption{Constraint language}
\label{fig:constraint}
\end{figure}

The simplest constraint language can be seen as that which corresponds to Haskell pattern matching. A constraint is a set of pattern matches, following the data structure given in figure \ref{fig:constraint}. A given data structure can be said to be a member of this constraint if there exists a match which would allow the data structure.

This given constraint language is not very powerful -- for example it is impossible to state in a finite space that all the elements of a list are the literal True. However, even with this limited constraint language the Risers example can be proven safe.

The analysis framework will be introduced using these constraints as a suitable example - but in  reality these constraints are not used. A Req is an expression and a constraint pair, stating that the expression must evaluate to something in the constraint. There are 3 operations which must be provided on every constraint implementation, whose signatures are given in figure n. Of these |(||>)| and |(<||)| are discussed in \S\ref{sec:backward}. The \T{notin} constraint is simpler, this generates a constructor which does \textit{not} match the constructor given. For example, an implementation for the above constraint language would be:

\begin{code}
notin c = map f (delete c (ctors c))
   where f x = Match x (replicate (arity c) Any)
\end{code}

An assumption made about the constraint system is that for any given finite type, there exist only a finite number of constraints. This simple constraint system violates property -- consider a recursive data structure such as a list, an infinite number of type-correct patterns are possible.

\subsection{Safety Preconditions}

\begin{figure}
\begin{code}
pre :: Expr -> Prop Req
pre (Var x         ) = True
pre (Path x _      ) = pre x
pre (Make _ xs     ) = and (map pre xs)
pre (Call f xs     ) = preFunc f xs && and (map pre xs)
pre (Case on alts  ) = and [f c e | (c,e) <- alts]
    where f c e = on :< notin c || pre e

preFunc :: FuncName -> [Expr] -> Prop Req
\end{code}
\caption{Precondition of an expression, \T{pre}}
\label{fig:precondition}
\end{figure}

There exists a precondition for every function such that if this precondition holds, then all function calls made will have their preconditions respected. The precondition on the \T{error} function is False. The process of analysis can be seen as deriving these preconditions.

A function is safe if the precondition on \T{main} is True. Given a function that returns the precondition on a functions arguments \T{preFunc}, a function to determine the precondition on an expression can be specified as \T{pre} in figure \ref{fig:precondition}. A fixed point is found iff:

\begin{code}
forall f, xs . preFunc f xs => pre (instantiate f xs)
\end{code}

This states that the preconditions assigned to each function by \T{preFunc} must imply the preconditions calculated on the instantiation of a function with its arguments. One such fixed point is that all calls to \T{preFunc} return False. A perfect precondition would be such that \T{preFunc f xs == pre (instantiate f xs)}.

The way preconditions are found for functions is that initially all functions are considered to have the precondition True, apart from error, which has the precondition False. For each precondition that has changed (initially error) all those functions which call this function have their bodies checked, assuming their current precondition. If the precondition has changed, then the new precondition is anded with the current one, and marked as changed. Once all preconditions are stable, a fixed point has been found.

The reason for anding with the previous value is to ensure termination -- the and ensures that the fixed point only ever becomes more restrictive, and therefore given a constraint model with a finite number of terms, is guaranteed to terminate.

A fixed point can be obtained more quickly by first propagating the functions at the leaf of the call tree. Once these have found a fixed point, it is only possible for the functions they call to require further computation with this expression. This is achieved by constructing a call graph, and taking a flattening to get an effective order for propagation. This can substantially reduce the amount of work required.


\subsection{Manipulating constraints}
\label{sec:backward}

Constraints on free variables can be considered as preconditions to the function they are scoped over. Constraints on function calls have to be treated specially, in a manner detailed in \S\ref{sec:template}, using a process called templating. All other constraints can be reduced using the process denoted by a function \T{back}, which takes a constraint and returns a predicate over constraints. This function is detailed in Figure~\ref{fig:backward}.

\begin{figure}
\begin{code}
back :: Req -> Prop Req
back (Var x         :< k) = Var x :< k
back (Path x p      :< k) = x :< (p |> k)
back (Make c xs     :< k) = Make c xs <| k
back (Call f xs     :< k) = template k f xs
back (Case on alts  :< k) = and [f c e | (c,e) <- alts]
    where f c e = on :< notin c || back (e :< k)

backs :: Req -> Prop Req
backs x = mapProp f (back x)
    where f (Var x :< k) = Var x :< k
          f x = backs x
\end{code}
\caption{Specification of backward analysis, \T{back}}
\label{fig:backward}
\end{figure}

\begin{description}

\item[The \T{Path} rule] moves the condition from the expression to the path. The |(||>)| operation can be seen as extending a constraint from being on one small part of the data structure, to being on the entire data structure. The implementation of this operation on the simple constraint type is now given:

\begin{code}
p |> k = map extend k
    where
    extend x = Match c (anys i ++ [x] ++ anys (n-i))
    anys j = replicate j Any
    i = pathPosn p
    c = pathCtor p
    n = arity c
\end{code}

\item[The \T{Make} rule] deals with an application of a constructor. The |(<||)| operator changes to the constraint from one on the entire structure, to one on each of the parts of the constructor.

\begin{code}
(Make c xs) <| k = or (map f k)
    where
    f Any = Any
    f (Match c2 xs2) = c2 == c &&
        and (zipWith (:<) xs (map (:[]) xs2))
\end{code}


\item[The \T{Case} rule] generates a conjunct for each alternative. The generated condition says either the subject of the case analysis has a different constructor (so this particular alternative is not executed in this circumstance), or the right hand side of the alternative is safe given the conditions for this expression.
\end{description}

The \T{backs} function repeatedly applies these rules until the only remaining constraints will be either on free variables, or function application. Function application is dealt with by templating, to reduce the problem to one on only free variables.

Once a constraint is on free variables, it is treated as a precondition to a function. It is then propagated to all the callers of the given function, and the process is repeated.

\subsection{Templating}
\label{sec:template}

When a constraint of the form \T{Call f xs :< k} is found, this needs to be reduced to a proposition of constraints on the expressions in \T{xs}. There are two strategies that can be used. One method is to use the $\beta$ substitution rule (\T{instantiate}) from $\lambda$-calculus, and replace the call with the body of \T{f}. Another alternative is to create fresh variables for each element in \T{xs}, solve this new problem, and then instantiate the results.

An earlier version of the Catch tool chose the former method, but the current version chooses the latter. The later is more amenable to finding a fixed point, and allows for a cache of the results to be built from which future questions can be answered. This cache of results can be seen as a set of axioms, and indeed this cache is saved to a file for future examination after Catch has finished.

The primary template mechanism, and its essential condition are given as:

\begin{code}
template :: Constraint -> FuncName -> [Expr] -> Prop Req

forall f, p xs => backs (instantiate f xs :< k)
    where p = template k f
\end{code}

The condition can be read as given a function call and a constraint, the resulting condition on the arguments must ensure that the condition holds on the instantiated body.

The mechanism used to calculate a fixed point is to execute the template function for one iteration, on the body of the function instantiated with free variables for each argument. After the result is obtained, it is anded with the existing value. If any other template is invoked within the template call then this must be considered at the same time as the current template. Each template is checked in turn until all templates do not change.

Given that there are a finite number of constraints, there are only a finite number of constraint/function pairs, which guarantees that this function terminates.


\section{Real Constraints}
\label{sec:constraint}

There are various interpretations of constraints, here I outline two that have been used in various versions of Catch. Neither is strictly more powerful than the other, both are capable of expressing constraints that the other system cannot.

\subsection{Semantics of Constraints}

The semantics of a constraint can be obtained by using the |(||>)| operator:

\begin{code}
data Value = Value CtorName [Value]

accept :: Value -> Constraint -> Bool
accept v k = backs (f v :< k)
    where f (Value c xs) = Make c (map f xs)
\end{code}

This assumes that the |(||>)| reduces a constraint to one only in terms of its inputs, which in all cases is true. This means that operator alone is responsible for determining the semantics of the constraint system.


\subsection{Regular Expression Based}

\begin{figure}
\begin{code}
data Constraint = RegExp :! [CtorName]

data RegExp = [Atom]

data Atom = Atom PathName
          | Star [PathName]

notin :: CtorName -> Constraint
notin c = [] :! delete c (ctors c)

(|>) :: PathName -> Constraint -> Constraint
p |> (r :! c) = integrate p r :! c

(<|) :: Expr -> Constraint -> Prop Req
(Make c xs) <| (r :! cs) =
    ewp r => c `elem` cs &&
    and (zipWith f (paths c) xs)
    where
    f p x = case differentiate p r of
                Nothing -> True
                Just r2 -> x :< (r2 :! cs)

ewp :: RegExp -> Bool
ewp x = all isStar x
   where isStar (Star _) = True
         isStar (Atom _) = False

integrate :: PathName -> RegExp -> RegExp
integrate p r | not (isRec p) = Atom p : r
integrate p (Star ps:r) | p `elem` ps = Star ps : r
                        | otherwise = Star (p:ps) : r
integrate p r = Star [p] : r

differentiate :: PathName -> Path -> Maybe Path
differentiate p [] = Nothing
differentiate p (Atom r:rs)
    | p == r = Just rs
    | otherwise = Nothing
differentiate p (Star r:rs)
    | if p `elem` r = Just (Star r:rs)
    | otherwise = differentiate p rs
\end{code}
\caption{Regular expression based constraints}
\label{fig:regexp}
\end{figure}

The original Catch tool used regular expression based constraints. A data type for the constraint, along with the essential operations upon it is given in Figure \ref{fig:regexp}. Given a constraint \T{(r :! c)}, \T{r} is referred to as the regular expression, and \T{c} is referred to as the set of constructors. Such a constraint matches those data values such that any well-defined application of a path of selectors described by $r$ must reach a constructor in the set $c$.

The meaning of a constraint is defined by:

\[ \tup{e,r,c} \Leftrightarrow (\forall l \in L(r) \bullet
\text{\textit{defined}}(e,l) \Rightarrow
\text{\textit{constructor}}(e\D{}l) \in c )
\]

Here $L(r)$ is the language represented by the regular expression $r$; \textit{defined} returns true if a path selection is well-defined; and \textit{constructor} gives the constructor used to create the data. Of course, since $L(r)$ is potentially infinite, this cannot be checked by enumeration.

If no path selection is well-defined then the constraint is vacuously true.

A regular expression is defined as:\\ \\
\begin{tabular}{ll}
$s+t$ & union of regular expressions $s$ and $t$ \\
$s\D t$ & concatenation of regular expressions $s$ then $t$ \\
$s\K$  & any number (possibly zero) occurrences of $s$ \\
\T{x} & a path, such as \T{hd} or \T{tl} \\
$\lambda$ & the language is the set containing the empty string \\
$\phi$ & the language is the empty set
\end{tabular}

The differentiation operation is that as defined by \citet{conway}, and called quotient in some text books. The empty word property (ewp) is equivalent to $\lambda \in L(r)$, and can be calculated simply from a regular expression. Integration is merely the inverse of differentiation -- not usually treated as a separate operation in most regular expression systems.

In the original version of Catch regular expressions were based on the full regular expression language. However the implementation above is limited to concatenation of atoms, or stars of unions. This restricted regular expression language, combined with $\phi$, is closed under integration and differentiation, making it particularly attractive as a reduced form. The $\phi$ alternative is catered for by the \T{Maybe} return type in the differentiation. The constraint \T{$\phi$ :! c} always evaluates to True, so in the |(<||)| Nothing is replaced by True.

The constructors, because of static typing and the restricted form of regular expression, must all by of the same type. There are some addition restrictions on regular expressions: all paths which are recursive must be under \T{Star} and all which are not must be \T{Atom}. Two Star's are not allowed in a row. With this restriction, and a finite type, this restriction is enough to ensure that there are only a finite number of regular expressions. Combined with the finite number of constructors, this is enough to guarantee finiteness, which satisfies the termination properties listed before.

\subsection{Predicates on constraints}

The expressions in constraints have five separate types, case statements, constructor applications, function applications, path expressions and variables. The variables are not actually free, but are quantified over the function in which they occur, who they are arguments to. As a result a complete constraint is expressed as $\forall f \bullet P(x)$, where $f$ is the function which provides scope for the free variables in $x$, and $P$ is a proposition with literal terms described as above.

From the definition of the constraints it is possible to construct a number of identities which can be used for simplification.

\begin{description}

\item[Path does not exist:] in the constraint \c{[]}{hd}{:} the expression \T{[]} does not have a \T{hd} path, so this constraint simplifies to true.

\item[Detecting failure:] the constraint \cc{[]}{:} simplifies to false because the \T{[]} value is not the constructor \T{:}.

\item[Empty path:] in the constraint $\tup{e,\phi,c}$, the regular expression is $\phi$, the empty language, so the constraint is always true.

\item[Exhaustive conditions:] in the constraint \cc{$e$}{:,[]} the condition lists all the possible constructors, if $e$ reaches weak head normal form then because of static typing $e$ must be one of these constructors, therefore this constraint simplifies to true.

\item[Algebraic conditions:] finally a couple of algebraic equivalences:

\[
\begin{array}{rcl}
\tup{e,r_1,c} \wedge \tup{e,r_2,c} & = & \tup{e,(r_1+r_2),c} \\
%incorrect, too conservative
%\tup{e,r,c_1} \vee   \tup{e,r,c_2} & = & \tup{e,r,c_1 \cup c_2} \\
\tup{e,r,c_1} \wedge \tup{e,r,c_2} & = & \tup{e,r,c_1 \cap c_2}
\end{array}
\]
\end{description}

\subsection{Binary Decision Diagrams}

BDD's - why they aren't right, negation in the BDD context.

Advantage of throwing away unused terms quickly.



\subsection{Enumeration Based}

These constraints are similar to those given in the example, but with infiniteness built in.

\subsection{Enumeration Based Propositions}

This is an extension over the enumeration based predicates.

\subsection{Comparison of Constraints}

Both constraint systems are capable of expressing a wide range of values. There are some areas where one is capabale of expressing different things.





\section{Haskell: Sugarless and Typeless}
\label{sec:transform}

The full Haskell language is a bit unwieldy for analysis. In particular the
syntactic sugar complicates analysis by introducing more types of expression to
consider. The checker works instead on a simplified language, a core to which
other Haskell programs can be reduced. This core language is a functional
language, making use of case expressions, function applications and algebraic
data types.

\subsection{Yhc Core}

In order to generate a simplified language, it is natural to start with a full
Haskell compiler, and we chose Yhc, a fork of nhc. The internal language of Yhc
is called PosLambda -- a simple variant of lambda calculus without types, but
with source position information. Yhc works by applying basic desugaring
transformations, without optimization transformations. This simplicity ensures
the generated PosLambda is close to the original Haskell in its structure. Each
top level function in a source file maps to a top level function in the
generated PosLambda, retaining the same name.

PosLambda does have some constructs that have no direct representation in
Haskell, for example there is a FatBar construct, which is used for compiling
pattern matches which require fall through behaviour. The PosLambda language
was always intended as an internal representation, and exposes certain details
that are specific to the compiler. We have introduced a new Core language to
Yhc, intended as a simple subset of Haskell where possible, on to which
PosLambda can easily be mapped. This Core language is not explicitly typed, and
has very few constructs. We have also written a library, Yhc.Core, which is
used by Yhc to generate these Core files, and by external programs to load and
manipulate the generated Core.

The importance of this Core language is not what remains, but what has been
removed.

\begin{itemize}
\item No syntactic sugar such as list comprehensions, \T{do} notation etc.
\item Only simple \T{case} statements, matching only the top level constructor
\item All \T{case} statements are complete, including an \T{error} call if
necessary
\item All names are fully qualified
\item Haskell's type classes have been removed (see \ref{sec:dict})
\item Only top level functions remain, all local functions have been lambda lifted
\item All constructor applications are fully saturated
\end{itemize}

\subsection{The Dictionary Transformation}

Most of the transformations in Yhc operate at a very local level, either on a
single function at a time or on a small expression within a function. The only
analysis/transformation phases which require information about more than one
function are type checking and the dictionary transformation, used to implement
type classes.

\yesexample

Take the following code:

\begin{code}
 f :: Eq a => a -> a -> Bool
 f x y = x == y || x /= y
\end{code}

is translated by Yhc into

\begin{code}
 f :: (a -> a -> Bool, a -> a -> Bool) -> a -> a -> Bool
 f dict x y = (||) (((==) dict) x y) (((/=) dict) x y)

 (==) (a,b) = a
 (/=) (a,b) = b
\end{code}

The \T{Eq} class consists of two functions, \T{(==)} and \T{(/=)}. Depending on
the type of the \T{a} variable in the function \T{f}, different code will be
executed for the two functions.

\noexample



The dictionary transformation generates a tuple, containing the functions to
use for the type class, and passes this structure into the \T{f} function as an
additional argument. The additional argument can be seen as the dictionary,
with functions being looked up in the dictionary to obtain the correct
behaviour at runtime.

The dictionary transformation is a global transformation. The \T{Eq} context in
\T{f} requires a dictionary to be accepted by \T{f}, and requires all the
callers of \T{f} to pass a dictionary as well.

The type class mechanism in Haskell allows certain classes to require other
classes, for example a definition of \T{Ord} requires a definition of \T{Eq}
for the same type. In response the dictionary transformation generates a nested
tuple, where the \T{Eq} dictionary is given as a member of the \T{Ord}
dictionary.

\subsection{First Order Haskell}

Having a simple Haskell-like language, there are essentially 2 features that
Haskell has that can be lived without, laziness and higher-order functions. It
is possible to convert Haskell to be a first order language by using
defunctionalization, and it is possible to convert Haskell to be a lazy
language by continuation passing. For the purposes of this checker, laziness is
a nice property, as it allows the code to be treated much like a set of maths
equations, without worrying about evaluation order etc. Higher order functions
are less helpful, and in fact obscure the flow of control within a program --
their removal is beneficial.

\subsection{Reynold's style defunctionalization}

Reynold's style defunctionalization is a simple method of generating a first
order program from a higher order one. Taking the following example:

\begin{code}
 map f x = case x of
                [] -> []
                (a:as) -> f a : map f as
\end{code}

Here f is a higher order input, of type \T{(a -> b)}. Defunctionalisation works
by creating a data type to represent all possible f's, and using that. For
example:

\begin{code}
 data Functions = Head | Tail

 apply Head x = head x
 apply Tail x = tail x

 map f x = case x of
     [] -> []
     (a:as) -> apply f a : map f as
\end{code}

Now all calls to map head would be replaced with map Head, and Head is a first
order value. This scheme can easily be extended to currying, let us take the
\T{add} function, which adds two \T{Int}'s:

\begin{code}
 data Functions = Head | Tail | Add0 | Add1 Int

 apply Add0 n = Add1 n
 apply (Add1 a) b = add a b
\end{code}

There are a couple of things to note about this approach. One is that while
this is still type safe, to do type checking would require a dependently typed
language. This is unfortunate, but for our checker (which does not use type
information), this is acceptable. The unacceptable aspect is the creation of an
apply function, whose semantics are very general. This essentially introduces a
bottleneck through which various properties must be proven. Asking questions,
such as is the result of apply a Cons or a Nil, are confusing.

As such, while Reynold's style defunctionalization is acceptable, it is not the
ideal method for removing higher order functions.

\subsection{Specialisation}

Often when a function is called, there is some information known about the
arguments -- for example one may be a constant. The way that the GHC program
makes use of this information is by inlining heavily, unfortunately sometimes
there is a good amount of information available, but the function is unsuitable
for inlining. Specialisation solves this problem neatly.

Let us consider for a minute the function \T{map}, defined as above. Now let us
consider the application \T{map f []}. Here it is not a great idea to inline
map, and GHC is unable to simplify this definition\footnote{GHC is able to
simplify the \T{map} provided in the standard Prelude using rewrite rules,
however by defining \T{map'} with the above definition, no simplification is
made}.

However, using specialisation, it is easy to generate \T{map} specialised with
\T{[]} as:

\begin{code}
 map f = []
\end{code}

Of course, now even a very conservative inline pass will succeed with this,
resulting in \T{[]} as the end result.

The notation used for specialisation is given as follows. The original version
of map takes two arguments, and is given as \T{map<?,?>} -- i.e. \T{map} with
two arguments, and no additional information. The version specialised on \T{[]}
is given as \T{map<?,[]>}. Note that here there is no second argument, the
specialised version wraps this up.

Now let us consider \T{map f (x:xs)}, this specialised in the same manner to
use the function:

\begin{code}
 map<?,?:?> f x xs = f x : map<?,?> xs
\end{code}

Note that the recursive case calls the standard map, as it cannot deduce any
information about xs. Obviously this has the potential for non-termination, say
the function \T{map<?,?:(?:(?: $\ldots$} was required. The way to make this
system terminating is to demand that the type must be new, for example it is
impossible to specialise the tail of a \T{(:)}, since the type of the tail is
the same as the initial list. This simple mechanism ensures termination, and at
the same time promotes good simplification of complex argument structures.

However, coming back to the original point, removing higher order functions.
These can be specialised in exactly the same way, for example \T{map head x}
can use the specialised version \T{map<head,?>}. Curried functions can be used
in the specialisation as well, \T{map<add ?,?>} is an example.

One particular usage of this specialisation treatment is the removal of the
dictionary transformation. For example, a function requiring Int's might be
specialised to \T{f<(intEq,intNeq),?,?>}. GHC and hbc both support a
specialisation annotation which achieves this effect, however full
specialisation performs this automatically.

There are of course disadvantages to specialisation, this is essentially a
whole program analysis transformation, and the performance is not stunning.
However, in practice it seems that the generated code corresponds much more
closely to what the original author had meant, and is significantly shorter --
which is not a massive surprise, Jones has shown this result in the specific
context of type classes before.




\section{Results}
\label{sec:results}

\subsection{Small Examples}

Map head, head reverse etc.


\subsection{Case Studies}

adjoxo, soda, clausify

\subsection{Nofib}

The whole nofib benchmark, in detail

Most of the benchmarks use [x] <- getArgs, this will always fail as getArgs is a primitive whose implementation is opaque to the system. As a result the system will always return False as the precondition to main.

The next step where many fail is that read is applied to an argument extracted from getArgs. Firstly this argument is entirely unknown. Secondly read is a sufficiently complicated function that while it can be modeled by Catch, there is no possibility for getting an appropriate abstraction which models the failure case. As a result, any program which calls read is unsafe, according to Catch. Using reads is perfectly sufficient, if the failure condition is appropriately handled.

Imaginery:

bernouilli - blows up going through firstify

digits-of-e1

digits-of-e2

exp3-8

gen-regexps

integrate

paraffins

primes

queens - fails in let elim (need to write abstract before let-elim)

rfib

tak

wheel-sieve1

wheel-sieve2

x2n1

\section{Related Work}
\label{sec:related}

Viewed as a \textbf{proof tool} this work can be seen as following Turner's goal to define a Haskell-like language which is total \cite{tfp:total}. Turner disallows incomplete pattern matches, saying this will ``force you to pay attention to exactly those corner cases which are likely to cause trouble''. Our checker may allow this restriction to be lifted, yet still retain a total programming language.

Viewed as a basic \textbf{pattern match checker}, the work on compiling warnings about incomplete and overlapping patterns is quite relevant \cite{ghc,pattern_match}. As noted in the introduction, these checks are only local.

Viewed as a \textbf{mistake detector} this tool has a similar purpose to the classic C Lint tool \cite{lint}, or Dialyzer \cite{dialyzer} -- a static checker for Erlang. The aim is to have a static checker that works on unmodified code, with no additional annotations. However, a key difference is that in Dialyzer all warnings indicate a genuine problem that needs to be fixed. Because Erlang is a dynamically typed language, a large proportion of Dialyzer's warnings relate to mistakes a type checker would have detected.

Viewed as a \textbf{soft type system} the checker can be compared to the tree automata work done on XML and XSL \cite{static_xslt}, which can be seen as an algebraic data type and a functional language. Another soft typing system with similarities is by Aiken \cite{type:dynamic}, on the functional language FL. This system tries to assign a type to each function using a set of constructors, for example \T{head} is given just \T{Cons} and not \T{Nil}.

My previous paper (not called Catch!) had lots of todo items on it, all have since been done

ESC/Haskell is fully manual.


\section{Conclusion}
\label{sec:conclusion}

I conclude that Catch is a useful tool.




% \appendix
% \section{Appendix Title}
%
% This is the text of the appendix, if you need one.

\acks

Thanks to the Plasma people, especially Matt and Tom for fruitful discussions.

\bibliographystyle{plainnat}
\bibliography{catch}



\end{document}
