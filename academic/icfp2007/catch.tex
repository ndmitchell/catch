\documentclass[preprint]{sigplanconf}

\usepackage{amsmath}
\usepackage{amssymb}
\usepackage{alltt}
\usepackage{url}
\usepackage{natbib}

% general stuff
\newcommand{\T}[1]{\texttt{#1}}
\newcommand{\tup}[1]{\ensuremath{\langle #1 \rangle}}
\newcommand{\mtxt}[1]{\textsf{#1}}

% examples
\newcounter{exmp}
\setcounter{exmp}{1}
\newcommand{\yesexample}{\subsubsection*{Example \arabic{exmp}}\addtocounter{exmp}{1}}
\newcommand{\noexample}{\hfill$\Box$}

\newcommand{\todo}[1]{\textbf{\textsc{Todo:} #1}}

% code blocks
\newenvironment{code}{\begin{alltt}\small}{\end{alltt}}
\newenvironment{codepage}
    {\begin{minipage}[h]{\textwidth}\begin{code}}
    {\end{code}\end{minipage}}

\newcommand{\K}{\ensuremath{^\ast}} % kleene star
\newcommand{\D}{\ensuremath{\cdot}} % central dot

\renewcommand{\c}[3]{\tup{\T{#1},\T{#2},\T{\{#3\}}}}
\newcommand{\cc}[2]{\c{#1}{$\lambda$}{#2}}

\newcommand{\s}[1]{\ensuremath{_{\tt #1}}} % subscript, in tt font
\newcommand{\g}[1]{\{#1\}} % group, put { } round it
\newcommand{\vecto}[1]{\overrightarrow{#1\;}}
\newcommand{\gap}{\;\;}


\begin{document}

\conferenceinfo{ICFP '07}{date, City.} %
\copyrightyear{2007} %
\copyrightdata{[to be supplied]}

\titlebanner{}        % These are ignored unless
\preprintfooter{Catch: A Technical Overview}   % 'preprint' option specified.

\title{Catch}
\subtitle{A Technical Overview}

\authorinfo{Neil Mitchell}
           {York}
           {ndm}
\authorinfo{Colin Runciman}
           {York}
           {colin}

\maketitle

\begin{abstract}
A Haskell program may fail at runtime with a pattern-match error if the program
has any incomplete (non-exhaustive) patterns in definitions or case
alternatives. This paper describes a static checker that allows non-exhaustive
patterns to exist, yet ensures that a pattern-match error does not occur. It
describes a constraint language that can be used to reason about pattern
matches, along with mechanisms to propagate these constraints between program
components.
\end{abstract}

% \category{CR-number}{subcategory}{third-level}

% \terms
% term1, term2

% \keywords
% keyword1, keyword2

\section{Introduction}
\label{sec:introduction}

Often it is useful to define pattern matches which are incomplete,
for example \T{head} fails on the empty list. Unfortunately programs
with incomplete pattern matches may fail at runtime.

Consider the following example:

\begin{code}
risers :: Ord a => [a] -> [[a]]
risers [] = []
risers [x] = [[x]]
risers (x:y:etc) = if x <= y then (x:s):ss else [x]:(s:ss)
    where (s:ss) = risers (y:etc)
\end{code}

A sample execution of this function would be:

\begin{code}
> risers [1,2,3,1,2]
[[1,2,3],[1,2]]
\end{code}

In the last line of the definition, \T{(s:ss)} is matched against
the output of \T{risers}. If \T{risers (y:etc)} returns an empty
list this would cause a pattern match error. It takes a few
moments to check this program manually -- and a few more to be
sure one has not made a mistake!

GHC \cite{ghc_manual} 6.4 has a warning flag to detect incomplete
patterns, which is named \T{-fwarn-incomplete-patterns}. Adding
this flag at compile time reports:\footnote{The additional flag
\T{-fwarn-simple-patterns} is needed, but this is due to GHC bug
number 1075259}

\begin{code}
Warning: Pattern match(es) are non-exhaustive
\end{code}

The Bugs (12.2.1) section of the manual notes that the checks are
sometimes wrong, particularly with string patterns or guards, and
that this part of the compiler ``needs an overhaul really''
\cite{ghc_manual}.

But the GHC checks are only local. If the function \T{head} is
defined, then it raises a warning. No effort is made to check the
\textit{callers} of \T{head} -- this is an obligation left to the
programmer.

Turning the \T{risers} function over to the checker developed in
this paper, the output is:

\begin{code}
> Safe
\end{code}

In addition the checker produces a set of axioms it has proved about Risers, along with an outline of the proof. Fuller details of how the checking is performed follow in Section 2.

\subsection{Road map}

This paper first introduces a constraint language, along with examples of
constraints and their informal meaning. Next we introduce the constraints in a
more formal way, along with a small core functional language. Rather than
extending the core language, we choose to transform full Haskell into this
reduced language -- this section details our method. Finally the system is
tested on a benchmark of Haskell programs, and the results are given.

The Catch tool can be seen as 3 entirely separate sections. Initially a program
is translated into the reduced Haskell language \S\ref{chap:yhc}. Second, the
program is transformed into a simpler first-order program with the same
semantics \S\ref{chap:defunc}. Third, a constraint language
\S\ref{chap:constraints} is used to analyse the program \S\ref{chap:backward}.
Results are presented \S\ref{chap:results} along with concluding remarks
\S\ref{chap:conc}.

\section{Walkthrough of Risers}

This section details all the stages that are executed in order to generate a proof that the \T{risers} function in the Introduction does not crash with a pattern match error. No one area is examined in detail, but an overall flavor is given.

\subsection{Modifying Risers for analysis}

There is one restriction on the functions that Catch can analyse -- they must be first order. In Haskell \citep{haskell}, because of the way type classes are implemented, this additionally means that they must not have type classes. The reason for this restriction is simple - with no knowledge about the input function the only conservative approximation is that all calls will give $\bot$.

Initially the type of risers is \T{risers :: Ord a => [a] -> [[a]]} - this can be easily fixed with.

\begin{code}
 main :: [Int] -> [[Int]]
 main x = risers x
\end{code}

\subsection{Conversion to a Core Language}

Rather than analyse full Haskell, the Catch tool analyses a first-order Core language, without lambda's (other than at the top level), partial application or let bindings. While this at first seems like a substantial restriction, a convertor is provided from the full Haskell 98 language to this very restricted language. Full details are provided in section n.

After converting to Core Haskell, and renaming to be more human readable, the resulting Core is as shown in Figure n. The function \T{risers4} and \T{risers2} correspond to the lazy pattern match in the \T{where}. The function \T{<=\#} is simply less than or equal, but specialised to the Int data type, unlike the standard one which is any type in the Num class.

\begin{figure}
\begin{code}
 risers3 x y = risers4 (risers (x : y))

 risers4 x = case x of
     (y:ys) -> (ys, y)
     [] -> error "Pattern Match Failure, 11:12."

 risers x = case x of
     [] -> []
     (y:ys) ->  case ys of
          [] -> (y : []) : []
          (z:zs) -> risers2 (risers3 z zs) (y <=# z) y

 risers2 x y z =  case y of
     True -> (z : snd x) : (fst x)
     False -> (z : []) : (snd x : fst x)
\end{code}
\caption{Risers in the Core language}
\label{fig:risers_core}
\end{figure}

\subsection{Analysis}

The analysis first starts by finding calls to \T{error}, then trying to prove that these calls will not be reached. In the above example it is easy to see there is only one error call, which corresponds to a pattern match error which has been desugared. From here the analysis determines that if risers4 is called, the argument must always be a : for the code to be safe.

Looking further back, this shows that risers3 calls risers4, passing (risers (x:y)) as the argument. In this particular case it is possible to see that risers of a : always results in a : - requiring in part the knowledge that risers2 is always a :. With these axioms it can be shown that the entire program is safe.

Unfortunately the analysis is not blessed with the foresight to know the answer. The analysis proceeds by searching for likely axioms, transforming constraints and otherwise manipulating program fragments until a fixed point is found. The exact transformation rules, along with the constraints that can be expressed by the system are the main substance of the analysis.


\section{The Data}

This section explains the underlying constraint system in Catch, focusing on how the constraints are put together to express properties of expressions and the data structures they evaluate to.

\subsection{Reduced expression language}

\begin{figure}
\begin{code}
data Expr = Var String
          | Sel Expr String
          | Make CtorName [Expr]
          | Call FuncName [Expr]
          | Case Expr [Alt]

data Alt = Alt CtorName Expr
         | AltDefault Expr
\end{code}
\caption{Core Data Type}
\label{fig:core}
\end{figure}

The abstract syntax tree for the reduced expression language is given in figure n.
The Core language chosen for this purpose is like most others, with the
exception of paths. The language is first order, has only simple case
statements, and only algebraic data types. The evaluation strategy is lazy. All
case statements are defined to be complete, with error being introduced where a
pattern match error would occur. This means that the challenge is checking for
calls to \textit{error}.

The one thing present in our Core language, but not in other languages, is the
\textit{path}. A path is represented as $e\D{}p$, where $e$ is the expression,
and $p$ is a \textit{selector}. Some examples of selectors include \T{hd} and
\T{tl}. A selector asserts that a data value has a given constructor, and then
follows its appropriate field. For example \T(1:2:[]\D{}hd) is \T{1}. It is
forbidden to write \T{[]\D{}hd}, or to ever require the evaluation of a
variables selector which has the wrong constructor.

Our use of paths contrasts with a traditional Core language's use of \T{case}
statements with bindings, however our choice has a number of merits:

\begin{enumerate}
\item Variables are only introduced at the function level, and have a scope of the entire function.
\item The rules to transform constraints are simplified.
\end{enumerate}

While the concept of Path's is different from normal, the benefits they provide are useful, allowing a simpler formulation of the rules.

\subsection{Constraints}

An expression ultimately evaluates to a (potentially infinite) data structure, or to $\bot{}$ caused by either non-termination of pattern match error. If an expression does evaluate to a data structure, then a constraint states the possible forms of data value it may take. A constraint $C$ is a set of data structures, potentially infinite.

Given a mechanism of expressing some set of constructors, $E(x) \in C$ states that the expression $x$ must, when evaluated be a member of the set defined by $C$. Since the Core language is typed, the constraint $C$ will only refer to a well typed term. These atomic constraints can be built up into a predicate of constraints, each possibly on different variables.

Since free variables are bound by function calls, it is possible to scope a constraint by making it a predicate. These can then be placed into a proposition. For example, $\forall f, i \in C$ states that for the function $f$, the argument $i$ must be in the set $C$. To take a concrete example, one particular example is $\forall risers4, x \in {:}$ - where ${:}$ is the set of values which start with a cons.

These constraints are sufficient to express many useful properties. Several underlying models of $C$ are possible, and will be discussed later.

\subsection{Manipulating constraints}

Constraints on free variables can be propagated by moving them from the callee to the caller. Constraints on function calls have to be treated specially. All other constraints can be reduced using the process denoted by a function $\varphi$, which takes a constraint and returns a predicate over constraints.
This function is detailed in Figure~\ref{fig:backward}.

In this figure, $C$ denotes a constructor, $c$ is a set of
constructors, $f$ is a function, $e$ is an expression, $r$ is a
regular expression over selectors and $s$ is a selector.

\begin{figure}

\renewcommand\theequation{sel}
\begin{equation}
    \tup{e\D{}s \in c} \rightarrow
    \tup{e,s\D{}r,c}
\end{equation}

\renewcommand\theequation{con}
\begin{equation}
\frac
    {
        P =
        \bigwedge_{i=1}^{\#\vecto{e}}
        \tup{e_i,\frac{\partial r}{\partial \mathcal{S}(C,i)},c}
    }
    {
        \tup{C \gap \vecto{e},r,c}
        \rightarrow (\lambda \in L(r) \Rightarrow C \in c) \wedge P
    }
\end{equation}

\renewcommand\theequation{cas}
\begin{equation}
\frac
    {
        P =
        \bigwedge_{i=1}^{\#\vecto{e}}
        (
            \tup{e_0,\lambda,C_i}
            \Rightarrow
            \tup{e_i,r,c}
        )
    }
    {
        \tup{\T{case } e_0 \T{ of \{}C_1 \T{->} e_1\T{;} \cdots
        \T{;} C_n \T{->} e_n \},r,c} \rightarrow P
    }
\end{equation}

\caption{Specification of backward analysis, $\varphi$} %
\label{fig:backward}
\end{figure}

\begin{description}

\item[The (sel) rule] moves the composition from the expression
to the path.

\item[The (con) rule] deals with an application of a constructor $C$.
If $\lambda$ is in the path language the $C$ must be permitted by
the condition. This depends on the \textit{empty word property}
(ewp) \cite{conway}, which can be calculated structurally on the
regular expression.

For each of the arguments to $C$, a new constraint is obtained from
the derivative of the regular expression with respect to that
argument's selector. This is denoted by $\partial r / \partial
\mathcal{S}(C,i)$, where $\mathcal{S}(C,i)$ gives the selector for
the $i$th argument of the constructor $C$. The differentiation
method is based on that described by Conway \cite{conway}. It can be
used to test for membership in the following way:

\[
\begin{array}{rcl}
\lambda \in L(r) & = & \text{ewp}(r) \\
s\D{}r' \in L(r) & = & r' \in L(\partial r / \partial s)
\end{array}
\]

Two particular cases of note are $\partial \lambda / \partial a =
\phi$ and $\partial \phi / \partial a = \phi$.

\item[The (app) rule] uses the notation $\mathcal{D}(f,\vecto{e})$ to
express the result of substituting each of the arguments in
$\vecto{e}$ into the body of the function $f$. The naive application
of this rule to any function with a recursive call will loop
forever. To combat this, if a function is already in the process of
being evaluated with the same constraint, its result is given as
true, and the recursive arguments are put into a special pile to be
examined later on, see \S\ref{sec:fixed_point} for details.

\item[The (cas) rule] generates a conjunct for each alternative.
The function $\mathcal{C}(C)$ returns the set of all other
constructors with the same result type as $C$, i.e.
$\mathcal{C}(\T{[]}) = \g{:}$. The generated condition says either
the subject of the case analysis has a different constructor (so
this particular alternative is not executed in this circumstance),
or the right hand side of the alternative is safe given the
conditions for this expression.
\end{description}


After these rules have been applied, the only remaining constraints will be
either on free variables, or function application. Function application is
dealt with by templating, to reduce the problem to one on only free variables.
Once a constraint is on free variables, it is propagated to all the callers of
the given function, and the process is repeated.



\section{Overview of Constraints}

In order to implement a checker that can ensure unfailing patterns,
it is useful to have some way of expressing properties of data
values. A constraint is written as $\tup{e,r,c}$ , where $e$ is an
expression, $r$ is a regular expression over selectors and $c$ is a
set of constructors. Such a constraint asserts that any well-defined
application to $e$ of a path of selectors described by $r$ must
reach a constructor in the set $c$.

These constraints are used as atoms in a predicate language with
conjunction and disjunction, so constraints can be about several
expressions and relations between them. The checker does not require
a negation operator. We also use the term constraint to refer to
logical formulae with constraints as atoms.

\yesexample

Consider the function \T{minimum}, defined as:

\begin{code}
minimum [x] = x
minimum (a:b:xs) = minimum (min a b : xs)

min a b = if a < b then a else b
\end{code}

Now consider the expression \T{minimum $e$}. The constraint that
must hold for this expression to be safe is \cc{$e$}{:}. This says
that the expression $e$ must reduce to an application of \T{:}, i.e.
a non-empty list. In this example the path was $\lambda$ -- the
empty path.\noexample

\yesexample

Consider the expression \T{map minimum $e$}. In this case the
constraint generated is \c{$e$}{tl\K\D hd}{:}. If we apply any
number (possibly zero) of \T{tl}s to $e$, then apply \T{hd}, we
reach a \T{:} construction. Values satisfying this constraint
include \T{[]} and \T{[[1],[2],[3]]}, but not \T{[[1],[]]}. The
value \T{[]} satisfies this constraint because it is impossible to
apply either \T{tl} or \T{hd}, and therefore the constraint does not
assert anything about the possible constructors.

\noexample

Constraints divide up into three parts -- the \textit{subject}, the
\textit{path} and the \textit{condition}.

\begin{description}
\item[The subject] in the above two examples was just $e$,
representing any expression -- including a call, a construction or
even a \T{case}.

\item[The path] is a regular expression over selectors.

A regular expression is defined as:\\ \\
\begin{tabular}{ll}
$s+t$ & union of regular expressions $s$ and $t$ \\
$s\D t$ & concatenation of regular expressions $s$ then $t$ \\
$s\K$  & any number (possibly zero) occurrences of $s$ \\
\T{x} & a selector, such as \T{hd} or \T{tl} \\
$\lambda$ & the language is the set containing the empty string \\
$\phi$ & the language is the empty set
\end{tabular}

\item[The condition] is a set of constructors which, due to static type
checking, must all be of the same result type.
\end{description}


\section{A Small Core Language}



\section{Manipulating Constraints}

\subsection{Generating Constraints}



\subsection{Manipulating Constraints}









\section{Haskell: Sugarless and Typeless}
\label{chap:yhc}

The full Haskell language is a bit unwieldy for analysis. In particular the
syntactic sugar complicates analysis by introducing more types of expression to
consider. The checker works instead on a simplified language, a core to which
other Haskell programs can be reduced. This core language is a functional
language, making use of case expressions, function applications and algebraic
data types.

\subsection{Yhc Core}

In order to generate a simplified language, it is natural to start with a full
Haskell compiler, and we chose Yhc, a fork of nhc. The internal language of Yhc
is called PosLambda -- a simple variant of lambda calculus without types, but
with source position information. Yhc works by applying basic desugaring
transformations, without optimization transformations. This simplicity ensures
the generated PosLambda is close to the original Haskell in its structure. Each
top level function in a source file maps to a top level function in the
generated PosLambda, retaining the same name.

PosLambda does have some constructs that have no direct representation in
Haskell, for example there is a FatBar construct, which is used for compiling
pattern matches which require fall through behaviour. The PosLambda language
was always intended as an internal representation, and exposes certain details
that are specific to the compiler. We have introduced a new Core language to
Yhc, intended as a simple subset of Haskell where possible, on to which
PosLambda can easily be mapped. This Core language is not explicitly typed, and
has very few constructs. We have also written a library, Yhc.Core, which is
used by Yhc to generate these Core files, and by external programs to load and
manipulate the generated Core.

The importance of this Core language is not what remains, but what has been
removed.

\begin{itemize}
\item No syntactic sugar such as list comprehensions, \T{do} notation etc.
\item Only simple \T{case} statements, matching only the top level constructor
\item All \T{case} statements are complete, including an \T{error} call if
necessary
\item All names are fully qualified
\item Haskell's type classes have been removed (see \ref{sec:dict})
\item Only top level functions remain, all local functions have been lambda lifted
\item All constructor applications are fully saturated
\end{itemize}

\subsection{The Dictionary Transformation}

Most of the transformations in Yhc operate at a very local level, either on a
single function at a time or on a small expression within a function. The only
analysis/transformation phases which require information about more than one
function are type checking and the dictionary transformation, used to implement
type classes.

\yesexample

Take the following code:

\begin{code}
 f :: Eq a => a -> a -> Bool
 f x y = x == y || x /= y
\end{code}

is translated by Yhc into

\begin{code}
 f :: (a -> a -> Bool, a -> a -> Bool) -> a -> a -> Bool
 f dict x y = (||) (((==) dict) x y) (((/=) dict) x y)

 (==) (a,b) = a
 (/=) (a,b) = b
\end{code}

The \T{Eq} class consists of two functions, \T{(==)} and \T{(/=)}. Depending on
the type of the \T{a} variable in the function \T{f}, different code will be
executed for the two functions.

\noexample

The dictionary transformation generates a tuple, containing the functions to
use for the type class, and passes this structure into the \T{f} function as an
additional argument. The additional argument can be seen as the dictionary,
with functions being looked up in the dictionary to obtain the correct
behaviour at runtime.

The dictionary transformation is a global transformation. The \T{Eq} context in
\T{f} requires a dictionary to be accepted by \T{f}, and requires all the
callers of \T{f} to pass a dictionary as well.

The type class mechanism in Haskell allows certain classes to require other
classes, for example a definition of \T{Ord} requires a definition of \T{Eq}
for the same type. In response the dictionary transformation generates a nested
tuple, where the \T{Eq} dictionary is given as a member of the \T{Ord}
dictionary.

\subsection{Case and Paths}

The Catch internal language has one significant difference from Yhc's Core
language, the use of \i{paths}. First we introduce a motivation for the path
construct, then we formalize its meaning. The full use of paths will only
become apparent later on, in the constraint section.

\yesexample

Let us take the function \T{head}, translated to the Core language of Yhc:

\begin{code}
 head x = case x of
              (a:as) -> a
              [] -> error "[] passed to head"
\end{code}

In this example, before the \T{case} statement is executed there is no
additional information about the value contained in \T{x}. Let us consider the
situation where it is known statically that \T{x} is a \T{(:)} constructor
before executing \T{head}. We can rewrite this code as:

\begin{code}
 headNonEmpty x = case x of
                      (a:as) -> a
\end{code}

We are now using the case expression for two different purposes: to select the
elements in a known data structure, and to test the type of a data structure at
runtime. Perhaps it would be convenient to have an alternative mechanism for
extracting fields from a value, as opposed to runtime flow of control choices.

The extra construct we introduce is the \i{path}. Informally we write
\T{x\D{}hd} to mean the first component of the \T{(:)}-constructed \T{x} value.
Now we can rewrite \T{headNonEmpty} as:

\begin{code}
 headNonEmpty x = x\(\D\)hd
\end{code}

With the addition of paths, we can now rewrite the standard \T{head} as:

\begin{code}
 head x = case x of
            (:) -> x\(\D\)hd
            [] -> error "[] passed to head"
\end{code}

\noexample

Now we have introduced paths, there is no need to specify names for each of the
fields given in a case alternative. If the \T{let} construct is removed as
well, then there is no possibility to introduce fresh variables within a
function -- only the initial arguments and top-level functions are in scope.

\subsubsection{Syntax and Semantics of Paths}

It is instructive to compare the differences and guarantees between Yhc's Core
language and that of Catch, especially in the area of \T{case} and paths. For
the \T{case} expressions, GHC's Core language can be considered the same as
Yhc's Core language.

In Yhc's Core there is only one construct that can demand evaluation, namely
the \T{case} statement. All \T{case} expressions must be complete, possibly
including a default alternative, but never leaving a possible value
unspecified. There is no way to extract a field out of a created value other
than by a \T{case} statement.

Within Catch there are now two different mechanisms for demanding the
evaluation of an expression, both path's and \T{case} expressions. A \T{case}
expression may still include a default alternative, but no longer has to be
complete. Only paths extract fields from values.

Formally, a path is defined as $\mtxt{path}(x,c,i)$, where $c$ is a constructor
and $n$ is an integer. If $n$ is the arity of constructor $c$ then $0 \leq i <
n$. The expression selects the $i$th field from the $c$ constructed value $x$.
The semantics in Haskell would be:

\begin{code}
 case \(x\) of
     \(c\) \(a\s{\textrm{1}}\ldots{}a\s{\textit{n}}\) -> \(a\s{\textit{i}}\)
     _ -> error "Ill formed path"
\end{code}

A sugared notation is commonly used instead of the $\textsf{path}$, where
\T{x\D{}hd} is equivalent to $\textsf{path}(\T{x},\T{:},0)$, \T{tl} is
equivalent to selecting the 1 index field. In a similar way, when a record with
uniquely named field selectors is given, these are used as the names. For the
purposes of this tool, a list is defined as:

\begin{code}
 data [] a = [] | (:) \{hd :: a, tl :: [] a\}
\end{code}

In much the same way as an object orientated language such as Java, the \D{}
operator associates to the left, so \T{x\D{}tl\D{}hd} is equivalent to
\T{(x\D{}tl)\D{}hd}.

There are no runtime semantics for an ill-formed path, they are statically
disallowed. Consider \T{[]\D{}hd}, this expression does not evaluate to
$\bot{}$ at runtime (as such a construct would in Haskell), but instead is
forbidden from being introduced. When transforming a \T{case} statement to use
paths instead, it is easy to see that the generated paths are well-formed --
the subject of the \T{case} is checked first. Any other transformation which
introduces a path must meet the proof obligation that this path is safe.

\section{First Order Haskell}
\label{chap:defunc}

Having a simple Haskell-like language, there are essentially 2 features that
Haskell has that can be lived without, laziness and higher-order functions. It
is possible to convert Haskell to be a first order language by using
defunctionalization, and it is possible to convert Haskell to be a lazy
language by continuation passing. For the purposes of this checker, laziness is
a nice property, as it allows the code to be treated much like a set of maths
equations, without worrying about evaluation order etc. Higher order functions
are less helpful, and in fact obscure the flow of control within a program --
their removal is beneficial.

\subsection{Reynold's style defunctionalization}

Reynold's style defunctionalization is a simple method of generating a first
order program from a higher order one. Taking the following example:

\begin{code}
 map f x = case x of
                [] -> []
                (a:as) -> f a : map f as
\end{code}

Here f is a higher order input, of type \T{(a -> b)}. Defunctionalisation works
by creating a data type to represent all possible f's, and using that. For
example:

\begin{code}
 data Functions = Head | Tail

 apply Head x = head x
 apply Tail x = tail x

 map f x = case x of
     [] -> []
     (a:as) -> apply f a : map f as
\end{code}

Now all calls to map head would be replaced with map Head, and Head is a first
order value. This scheme can easily be extended to currying, let us take the
\T{add} function, which adds two \T{Int}'s:

\begin{code}
 data Functions = Head | Tail | Add0 | Add1 Int

 apply Add0 n = Add1 n
 apply (Add1 a) b = add a b
\end{code}

There are a couple of things to note about this approach. One is that while
this is still type safe, to do type checking would require a dependently typed
language. This is unfortunate, but for our checker (which does not use type
information), this is acceptable. The unacceptable aspect is the creation of an
apply function, whose semantics are very general. This essentially introduces a
bottleneck through which various properties must be proven. Asking questions,
such as is the result of apply a Cons or a Nil, are confusing.

As such, while Reynold's style defunctionalization is acceptable, it is not the
ideal method for removing higher order functions.

\subsection{Specialisation}

Often when a function is called, there is some information known about the
arguments -- for example one may be a constant. The way that the GHC program
makes use of this information is by inlining heavily, unfortunately sometimes
there is a good amount of information available, but the function is unsuitable
for inlining. Specialisation solves this problem neatly.

Let us consider for a minute the function \T{map}, defined as above. Now let us
consider the application \T{map f []}. Here it is not a great idea to inline
map, and GHC is unable to simplify this definition\footnote{GHC is able to
simplify the \T{map} provided in the standard Prelude using rewrite rules,
however by defining \T{map'} with the above definition, no simplification is
made}.

However, using specialisation, it is easy to generate \T{map} specialised with
\T{[]} as:

\begin{code}
 map f = []
\end{code}

Of course, now even a very conservative inline pass will succeed with this,
resulting in \T{[]} as the end result.

The notation used for specialisation is given as follows. The original version
of map takes two arguments, and is given as \T{map<?,?>} -- i.e. \T{map} with
two arguments, and no additional information. The version specialised on \T{[]}
is given as \T{map<?,[]>}. Note that here there is no second argument, the
specialised version wraps this up.

Now let us consider \T{map f (x:xs)}, this specialised in the same manner to
use the function:

\begin{code}
 map<?,?:?> f x xs = f x : map<?,?> xs
\end{code}

Note that the recursive case calls the standard map, as it cannot deduce any
information about xs. Obviously this has the potential for non-termination, say
the function \T{map<?,?:(?:(?: $\ldots$} was required. The way to make this
system terminating is to demand that the type must be new, for example it is
impossible to specialise the tail of a \T{(:)}, since the type of the tail is
the same as the initial list. This simple mechanism ensures termination, and at
the same time promotes good simplification of complex argument structures.

However, coming back to the original point, removing higher order functions.
These can be specialised in exactly the same way, for example \T{map head x}
can use the specialised version \T{map<head,?>}. Curried functions can be used
in the specialisation as well, \T{map<add ?,?>} is an example.

One particular usage of this specialisation treatment is the removal of the
dictionary transformation. For example, a function requiring Int's might be
specialised to \T{f<(intEq,intNeq),?,?>}. GHC and hbc both support a
specialisation annotation which achieves this effect, however full
specialisation performs this automatically.

There are of course disadvantages to specialisation, this is essentially a
whole program analysis transformation, and the performance is not stunning.
However, in practice it seems that the generated code corresponds much more
closely to what the original author had meant, and is significantly shorter --
which is not a massive surprise, Jones has shown this result in the specific
context of type classes before.

\section{Detecting Match Errors}

Detecting pattern match errors is equivalent to doing reachability analysis for
the \T{error} function, since Yhc transforms potential pattern match errors to
calls to \T{error}. It is usually very easy at a local level to determine
whether a pattern match will occur, for example with \T{head}, the precondition
for the function to be safe is that the input argument must be a \T{(:)}. What
is hard to determine is whether the input to head is a \T{(:)} or not. Backward
analysis is the process by which this is determined.

Backward analysis works by taking a condition on an expression, and
transforming it to a condition on the free variables in that expression.
Formally this is denoted by:

\[ P'(\text{fv}(x)) \Rightarrow P(x) \]

The process of backward analysis takes $x$ and $P$, and generates $P'$, where
$\text{fv}(x)$ denotes the free variables in $x$. The implication is used
rather than equals because the generated condition has to be sufficient to
ensure the original condition, but may be more strict.

So now let us consider the expression, \T{head (f x)}. The backwards analysis
will convert the condition on \T{f x}, namely that it must be a \T{(:)}, to a
condition on \T{x}. For example, if \T{f} is equivalent to \T{id}, the
identity, then the condition on \T{x} would be that it is a \T{(:)}. However,
if \T{f} was \T{(:[])} -- the function that puts an element into a single
element list -- then the precondition would be $True$, i.e. always safe.

Using this scheme it is necessary to have some formalism for $P$, some
constraint language. The backwards analysis function must also be defined. In
the following two sections, such a language is presented, along with a
backwards analysis function.


\section{A constraint language}
\label{chap:constraints}

Semantics

\subsection{An atomic constraint}

In order to implement a checker that can ensure unfailing patterns,
it is useful to have some way of expressing properties of data
values. A constraint is written as $\tup{e,r,c}$ , where $e$ is an
expression, $r$ is a regular expression over selectors and $c$ is a
set of constructors. Such a constraint asserts that any well-defined
application to $e$ of a path of selectors described by $r$ must
reach a constructor in the set $c$.

These constraints are used as atoms in a predicate language with
conjunction and disjunction, so constraints can be about several
expressions and relations between them. The checker does not require
a negation operator. We also use the term constraint to refer to
logical formulae with constraints as atoms.

\yesexample

Consider the function \T{minimum}, defined as:

\begin{code}
minimum xs = case xs of
                  [x]      -> x
                  (a:b:xs) -> minimum (min a b : xs)

min a b = case a < b of
               True  -> a
               False -> b
\end{code}

Now consider the expression \T{minimum $e$}. The constraint that
must hold for this expression to be safe is \cc{$e$}{:}. This says
that the expression $e$ must reduce to an application of \T{:}, i.e.
a non-empty list. In this example the path was $\lambda$ -- the
empty path.\noexample

\yesexample

Consider the expression \T{map minimum $e$}. In this case the
constraint generated is \c{$e$}{tl\K\D hd}{:}. If we apply any
number (possibly zero) of \T{tl}s to $e$, then apply \T{hd}, we
reach a \T{:} construction. Values satisfying this constraint
include \T{[]} and \T{[[1],[2],[3]]}, but not \T{[[1],[]]}. The
value \T{[]} satisfies this constraint because it is impossible to
apply either \T{tl} or \T{hd}, and therefore the constraint does not
assert anything about the possible constructors.

\noexample

Constraints divide up into three parts -- the \textit{subject}, the
\textit{path} and the \textit{condition}.

\begin{description}
\item[The subject] in the above two examples was just $e$,
representing any expression -- including a call, a construction or
even a \T{case}.

\item[The path] is a regular expression over selectors.

A regular expression is defined as:\\ \\
\begin{tabular}{ll}
$s+t$ & union of regular expressions $s$ and $t$ \\
$s\D t$ & concatenation of regular expressions $s$ then $t$ \\
$s\K$  & any number (possibly zero) occurrences of $s$ \\
\T{x} & a selector, such as \T{hd} or \T{tl} \\
$\lambda$ & the language is the set containing the empty string \\
$\phi$ & the language is the empty set
\end{tabular}

\item[The condition] is a set of constructors which, due to static type
checking, must all be of the same result type.
\end{description}

The meaning of a constraint is defined by:

\[ \tup{e,r,c} \Leftrightarrow (\forall l \in L(r) \bullet
\text{\textit{defined}}(e,l) \Rightarrow
\text{\textit{constructor}}(e\D{}l) \in c )
\]

\noindent Here $L(r)$ is the language represented by the regular
expression $r$; \textit{defined} returns true if a path selection is
well-defined; and \textit{constructor} gives the constructor used to
create the data. Of course, since $L(r)$ is potentially infinite,
this cannot be checked by enumeration.

If no path selection is well-defined then the constraint is
vacuously true.

\subsection{Simplifying the Constraints}
\label{sec:constraint_simplify}

From the definition of the constraints it is possible to construct a
number of identities which can be used for simplification.

\begin{description}

\item[Path does not exist:] in the constraint \c{[]}{hd}{:} the expression
\T{[]} does not have a \T{hd} path, so this constraint simplifies to
true.

\item[Detecting failure:] the constraint \cc{[]}{:} simplifies to false
because the \T{[]} value is not the constructor \T{:}.

\item[Empty path:] in the constraint $\tup{e,\phi,c}$, the
regular expression is $\phi$, the empty language, so the constraint
is always true.

\item[Exhaustive conditions:] in the constraint \cc{$e$}{:,[]}
the condition lists all the possible constructors, if $e$ reaches
weak head normal form then because of static typing $e$ must be one
of these constructors, therefore this constraint simplifies to true.

\item[Algebraic conditions:] finally a couple of algebraic equivalences:

\[
\begin{array}{rcl}
\tup{e,r_1,c} \wedge \tup{e,r_2,c} & = & \tup{e,(r_1+r_2),c} \\
%incorrect, too conservative
%\tup{e,r,c_1} \vee   \tup{e,r,c_2} & = & \tup{e,r,c_1 \cup c_2} \\
\tup{e,r,c_1} \wedge \tup{e,r,c_2} & = & \tup{e,r,c_1 \cap c_2}
\end{array}
\]
\end{description}

\subsection{Predicates on constraints}

The expressions in constraints have five separate types, case statements,
constructor applications, function applications, path expressions and
variables. The variables are not actually free, but are quantified over the
function in which they occur, who they are arguments to. As a result a complete
constraint is expressed as $\forall f \bullet P(x)$, where $f$ is the function
which provides scope for the free variables in $x$, and $P$ is a proposition
with literal terms described as above.

\subsection{Restricted Regular Expressions}

Restricted according to type.

\section{Backward Analysis Function}
\label{chap:backward}

Given the semantics for Haskell, along with the semantics for the constraint
system it is relatively easy to derive the backward analysis function on all
the syntactic elements, excluding function application. The results for these
are now presented, along with commentary:

\subsection{The Initial Constraints}

In general, a \T{case} expression, where $\vecto{v}$ are the
arguments to a constructor:

\begin{code}
 case \(e\) of
     C\s{1} -> val\s{1}
     \(\cdots\)
     C\s{n} -> val\s{n}
\end{code}

\noindent produces the initial constraint
\cc{$e$}{C\s{1},\ldots,C\s{n}}. If the case alternatives are
exhaustive, then this can be simplified to true. All \T{case}
expressions in the program are found, their initial constraints are
found, and these are joined together with conjunction.

\subsection{Transforming the constraints}

For each constraint in turn, if the subject is \T{x\s{f}} (i.e. the
\T{x} argument to \T{f}), the checker searches for every application
of \T{f}, and gets the expression for the argument \T{x}. On this
expression, it sets the existing constraint. This constraint is then
transformed using a backward analysis (see \S\ref{sec:backward}),
until a constraint on arguments is found.

\yesexample

Consider the constraint \cc{xs\s{minimum}}{:} -- that is
\T{minimum}'s argument \T{xs} must be a non-empty list. If the
program contains the expression:

\begin{code}
f x = minimum (g x)
\end{code}

\noindent then the derived constraint is \cc{(g x\s{f})}{:}.
\noexample

\subsection{Backward Analysis}
\label{sec:backward}

Backward analysis takes a constraint in which the subject is a
compound expression, and derives a combination of constraints over
arguments only. This process is denoted by a function $\varphi$,
which takes a constraint and returns a predicate over constraints.
This function is detailed in Figure~\ref{fig:backward}.

In this figure, $C$ denotes a constructor, $c$ is a set of
constructors, $f$ is a function, $e$ is an expression, $r$ is a
regular expression over selectors and $s$ is a selector.

\begin{figure}

\renewcommand\theequation{sel}
\begin{equation}
    \tup{e\D{}s,r,c} \rightarrow
    \tup{e,s\D{}r,c}
\end{equation}

\renewcommand\theequation{con}
\begin{equation}
\frac
    {
        P =
        \bigwedge_{i=1}^{\#\vecto{e}}
        \tup{e_i,\frac{\partial r}{\partial \mathcal{S}(C,i)},c}
    }
    {
        \tup{C \gap \vecto{e},r,c}
        \rightarrow (\lambda \in L(r) \Rightarrow C \in c) \wedge P
    }
\end{equation}

\renewcommand\theequation{cas}
\begin{equation}
\frac
    {
        P =
        \bigwedge_{i=1}^{\#\vecto{e}}
        (
            \tup{e_0,\lambda,C_i}
            \Rightarrow
            \tup{e_i,r,c}
        )
    }
    {
        \tup{\T{case } e_0 \T{ of \{}C_1 \T{->} e_1\T{;} \cdots
        \T{;} C_n \T{->} e_n \},r,c} \rightarrow P
    }
\end{equation}

\caption{Specification of backward analysis, $\varphi$} %
\label{fig:backward}
\end{figure}

\begin{description}

\item[The (sel) rule] moves the composition from the expression
to the path.

\item[The (con) rule] deals with an application of a constructor $C$.
If $\lambda$ is in the path language the $C$ must be permitted by
the condition. This depends on the \textit{empty word property}
(ewp) \cite{conway}, which can be calculated structurally on the
regular expression.

For each of the arguments to $C$, a new constraint is obtained from
the derivative of the regular expression with respect to that
argument's selector. This is denoted by $\partial r / \partial
\mathcal{S}(C,i)$, where $\mathcal{S}(C,i)$ gives the selector for
the $i$th argument of the constructor $C$. The differentiation
method is based on that described by Conway \cite{conway}. It can be
used to test for membership in the following way:

\[
\begin{array}{rcl}
\lambda \in L(r) & = & \text{ewp}(r) \\
s\D{}r' \in L(r) & = & r' \in L(\partial r / \partial s)
\end{array}
\]

Two particular cases of note are $\partial \lambda / \partial a =
\phi$ and $\partial \phi / \partial a = \phi$.

\item[The (app) rule] uses the notation $\mathcal{D}(f,\vecto{e})$ to
express the result of substituting each of the arguments in
$\vecto{e}$ into the body of the function $f$. The naive application
of this rule to any function with a recursive call will loop
forever. To combat this, if a function is already in the process of
being evaluated with the same constraint, its result is given as
true, and the recursive arguments are put into a special pile to be
examined later on, see \S\ref{sec:fixed_point} for details.

\item[The (cas) rule] generates a conjunct for each alternative.
The function $\mathcal{C}(C)$ returns the set of all other
constructors with the same result type as $C$, i.e.
$\mathcal{C}(\T{[]}) = \g{:}$. The generated condition says either
the subject of the case analysis has a different constructor (so
this particular alternative is not executed in this circumstance),
or the right hand side of the alternative is safe given the
conditions for this expression.
\end{description}


After these rules have been applied, the only remaining constraints will be
either on free variables, or function application. Function application is
dealt with by templating, to reduce the problem to one on only free variables.
Once a constraint is on free variables, it is propagated to all the callers of
the given function, and the process is repeated.

\subsection{Templating}

When a constraint of the form \tup{f \overrightarrow{e},r,c} is
found, there are two choices which at first glance can be used. One
method is to use the $\beta$ substitution rule from
$\lambda$-calculus, and replace $f$ with its body. Another
alternative is to create fresh variables for each element in
$\overrightarrow{e}$, solve this new problem, and then instantiate
the results.

The original Catch program as described in TFP chose the former method, the
program described in this paper chooses the later. The advantage of the first
method is that it very naturally models the $\lambda$-calculus, and therefore
has an intuative feel. The disadvantage is that to acheive a fixed point on the
template instantiation requires some kind fixed point on expressions. By moving
to the second method the algorithm becomes substaintially simpler. It also has
the advantage that the free variables can be chosen in a way such that it is
likely to generate the same instantiation again, and can be satisfied out of a
cache in future invocations. The addition of a cache turns out to be a
substantial time benefit, often changing the underlying complexity of the
algorithm.

The templating method is as follows:

Given a constraint \tup{f \overrightarrow{e},r,c}, the constraint %
\tup{f v_0\ldots v_{\sharp \overrightarrow{e},r,c}} is generated and
solved with simple $\beta$ expansion. This method continues as
constraints are generated, but with the knowledge that this
identical constraint gives $True$ as its generated condition within
this subcomputation. After having solved the constraint, the result
is compared with $True$. If the result matches, it is used, if it
does not then a new computation is attempted with the assumption
that the result matches the first result. This process continues
until a fixed point is found, which is returned as the result. By
applying $\wedge$ at each stage to the previous answer, it is easy
to guarantee that the fixed point found is the greatest fixed point.

The implementation method for this is to use a stack of variables, initially
when a call is encountered the value $True$ and the name of the function and
its constraint are pushed onto the stack. During the computation if a function
which is not recognised is reached, it too is pushed on the stack. However, if
an existing function is spotted, then the value already on the stack is used.
After a computation is finished, is the value matches that which was already on
the stack then a fixed point is found, otherwise the process starts again with
a new initial value.

If a function does not call any functions which are already on the stack, then
its result is not effected by the current stack, and therefore the end result
can be cached. This is of particular importance as it allows many sections of
the computation to be solved once, even though the fixed point may traverse
over that function many times.

\subsection{Propagation}

Once a constraint has been reduced to a proposition of constraints only on free
variables, it is possible to propagate these constraints to the caller of the
function whose free variables are being examined. A fixed point to this process
can be achieved by using $\wedge$ on each iteration, and stopping when no
functions change.

A fixed point can be obtained more quickly by first propagating the functions
at the leaf of the call tree. Once these have found a fixed point, it is only
possible for the functions they call to require further computation with this
expression. This is achieved by constructing a call graph, and taking a
flattening to get an effective order for propagation. This can substantially
reduce the amount of work required.

\section{Implementation concerns}

\subsection{Function guards vs case statements}

While the code operates on case statements, you can flatten them and achieve
better results.

MCase's for easier code, sort of like pre-evaluation of some stuff.

\subsection{Binary Decision Diagrams}

BDD's - why they aren't right, negation in the BDD context.

Advantage of throwing away unused terms quickly.

\section{Results}
\label{chap:results}

\subsection{Small Examples}

Map head, head reverse etc.


\subsection{Case Studies}

adjoxo, soda, clausify

\subsection{Nofib}

The whole nofib benchmark, in detail

\section{Conclusion}
\label{chap:conc}






\appendix
\section{Appendix Title}

This is the text of the appendix, if you need one.

\acks

Acknowledgments, if needed.

\bibliographystyle{plainnat}
\bibliography{catch}



\end{document}
