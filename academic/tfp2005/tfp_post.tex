\documentclass[book]{tfp05symp}

% ! LaTeX Error: Math version `bold' is not defined.
% \usepackage{amsmath}
% \usepackage{amssymb}
%\usepackage{stmaryrd}
%\usepackage{latexsym}

\usepackage{amstext}
% \usepackage{listings}
\usepackage{url}
\usepackage{alltt}
\usepackage{comment}

% \lstset{breaklines, showstringspaces=false,
% basicstyle=\ttfamily\small, keywordstyle=\tt\bf, commentstyle=\it,
% columns=fullflexible, mathescape=true}


\newcommand{\T}[1]{\texttt{#1}}
\newcommand{\Tu}[1]{\lstinline{#1}}
\newcommand{\gap}{\;\;}
\newcommand{\tup}[1]{\ensuremath{\langle #1 \rangle}}

\renewcommand{\c}[3]{\tup{\T{#1},\T{#2},\T{\{#3\}}}}
\newcommand{\cc}[2]{\c{#1}{$\lambda$}{#2}}

% subscript, in tt font
\newcommand{\s}[1]{\ensuremath{_{\tt #1}}}
% group, put { } round it
\newcommand{\g}[1]{\{#1\}}

% kleene star
\newcommand{\K}{\ensuremath{^\ast}}
% central dot
\newcommand{\D}{\ensuremath{\cdot}}

\newcounter{todo}
\setcounter{todo}{1}
\newcommand{\todo}[1]{\textbf{\textsc{Todo \arabic{todo}:}} #1 \addtocounter{todo}{1}}

\newcommand{\vecto}[1]{\overrightarrow{#1\;}}

% grrrr to big levels
\newcommand{\arrleft}{
    \begin{picture}(15,6)
    \put(0,0){<}
    \put(0.8,0){-}
    \put(2,0){-}
    \put(4,0){-}
    \put(6,0){-}
    \put(8,0){-}
    \put(10,0){-}
    \put(12,0){-}
    \put(14.1,0){|}
    % \put(0.5,3){\line(1,0){10}}
    % \put(10,0){\line(0,1){6}}
    \end{picture}
    }

\newcommand{\mapsfrom}{\hookleftarrow}

\newcommand{\boxxsize}{7}

\newcommand{\boxx}{
    \begin{picture}(\boxxsize,\boxxsize)
    \put(0,0){\line(0,1){\boxxsize}}
    \put(\boxxsize,0){\line(0,1){\boxxsize}}
    \put(0,\boxxsize){\line(1,0){\boxxsize}}
    \put(0,0){\line(1,0){\boxxsize}}
    \end{picture}
    }
% \begin{picture}(340,070)
%
% \put(000,050){\framebox(120,20){\small{Find non-exhaustive patterns}}}
% \put(140,050){\framebox(80,20){\small{Find callers}}}
% \put(140,010){\makebox(80,20){\small{Backward analysis}}}
% \put(140,000){\makebox(80,20){\small{Fixed pointing}}}
% \put(240,000){\framebox(80,30){\small{Report result}}}
% \put(140,000){\framebox(80,30){ }}
%
% \put(120,060){\vector(1,0){20}}
% \put(160,050){\vector(0,-1){20}}
% \put(200,030){\vector(0,01){20}}
% \put(220,015){\vector(1,0){20}}


\newcounter{exmp}
\setcounter{exmp}{1}
\newcommand{\yesexample}{\subsubsection*{Example \arabic{exmp}}\addtocounter{exmp}{1}}
\newcommand{\noexample}{\hfill\boxx}

\newenvironment{code}{\begin{alltt}\small}{\end{alltt}}
\newenvironment{codepage}
    {\begin{minipage}[h]{\textwidth}\begin{code}}
    {\end{code}\end{minipage}}




\begin{document}

\chaptitle{A static checker for safe pattern matching in Haskell}

\chapter{A Static Checker for Safe Pattern Matching in Haskell}

\chapauthors{Neil Mitchell and Colin Runciman \footnotemark[1]{}}

\footnotetext[1]{University of York, UK.
http://www.cs.york.ac.uk/$\sim$ndm and
http://www.cs.york.ac.uk/$\sim$colin}

\addtocounter{footnote}{1}

\chapauthorstoc{Neil Mitchell and Colin Runciman}


% \title{Unfailing Haskell: \\A Static Checker for Pattern Matching}

% \author{Neil Mitchell \and Colin Runciman}
%            {University of York, UK}
%            {http://www.cs.york.ac.uk/$\sim$ndm}
% \authorinfo{Colin Runciman}
%           {University of York, UK}
%           {http://www.cs.york.ac.uk/$\sim$colin}

% \author{Neil Mitchell \and Colin Runciman \\ \footnotesize{http://www.cs.york.ac.uk/$\sim$ndm , http://www.cs.york.ac.uk/$\sim$colin}}
% $ for textpad's syntax hilighting
% \institute{University of York, UK}

% \maketitle

\paragraph{Abstract}
% \begin{abstract}
A Haskell program may fail at runtime with a pattern-match error if
the program has any incomplete (non-exhaustive) patterns in
definitions or case alternatives. This paper describes a static
checker that allows non-exhaustive patterns to exist, yet ensures
that a pattern-match error does not occur. It describes a constraint
language that can be used to reason about pattern matches, along
with mechanisms to propagate these constraints between program
components.
%\end{abstract}



\section{Introduction}
\label{sec:introduction}

Often it is useful to define pattern matches which are incomplete,
for example \T{head} fails on the empty list. Unfortunately programs
with incomplete pattern matches may fail at runtime.

Consider the following example:

\begin{code}
risers :: Ord a => [a] -> [[a]]
risers [] = []
risers [x] = [[x]]
risers (x:y:etc) = if x <= y then (x:s):ss else [x]:(s:ss)
    where (s:ss) = risers (y:etc)
\end{code}

A sample execution of this function would be:

\begin{code}
> risers [1,2,3,1,2]
[[1,2,3],[1,2]]
\end{code}

In the last line of the definition, \T{(s:ss)} is matched against
the output of \T{risers}. If \T{risers (y:etc)} returns an empty
list this would cause a pattern match error. It takes a few
moments to check this program manually -- and a few more to be
sure one has not made a mistake!

GHC \cite{ghc_manual} 6.4 has a warning flag to detect incomplete
patterns, which is named \T{-fwarn-incomplete-patterns}. Adding
this flag at compile time reports:\footnote{The additional flag
\T{-fwarn-simple-patterns} is needed, but this is due to GHC bug
number 1075259}

\begin{code}
Warning: Pattern match(es) are non-exhaustive
\end{code}

\begin{comment}
The Bugs (12.2.1) section of the manual notes that the checks are
sometimes wrong, particularly with string patterns or guards, and
that this part of the compiler ``needs an overhaul really''
\cite{ghc_manual}.
\end{comment}

But the GHC checks are only local. If the function \T{head} is
defined, then it raises a warning. No effort is made to check the
\textit{callers} of \T{head} -- this is an obligation left to the
programmer.

Turning the \T{risers} function over to the checker developed in
this paper, the output is:

\begin{code}
> (risers (y:etc))\g{:}
> True
\end{code}

The checker first decides that for the code to be safe the recursive
call to \T{risers} must always yield a non-empty list. It then
notices that if the argument in a \T{risers} application is
non-empty, then so will the result be. This satisfies it, and it
returns True, guaranteeing that no pattern-match errors will occur.

\subsection{Roadmap}

This paper starts by introducing a reduced language similar to
Haskell in \S\ref{sec:reduced_haskell}. Next a constraint language
is introduced in \S\ref{sec:constraints} and algorithms are given to
manipulate these constraints in \S\ref{sec:determining}. A worked
example is given in \S\ref{sec:worked_example}, followed by a range
of small examples and a case study in \S\ref{sec:examples}. This
paper is compared to related work in \S\ref{sec:related_work}.
Finally conclusions are given in \S\ref{sec:conclusions}, along with
some remaining tasks -- this paper reports on work in progress.

\section{Reduced Haskell}
\label{sec:reduced_haskell}

The full Haskell language is a bit unwieldy for analysis. In
particular the syntactic sugar complicates analysis by introducing
more types of expression to consider. The checker works instead on a
simplified language, a core to which other Haskell programs can be
reduced. This core language is a functional language, making use of
case expressions, function applications and algebraic data types.

As shown in example 1, only one defining equation per function is
permitted, pattern-matching occurs only in case expressions and
every element within a constructor must be uniquely named by a
selector (e.g. \T{hd} and \T{tl}). A convertor from a reasonable
subset of Haskell to this reduced language has been written.

\yesexample

\begin{code}
data [] a = (:) \{hd :: a, tl :: [] a\} | []

head x = case x of (a:_) -> a

map f xs = case xs of
                []     -> []
                (a:as) -> f x : map f as

reverse xs = rev xs []

reverse2 x a = case x of
                    []     -> a
                    (y:ys) -> reverse2 ys (y:a)\noexample
\end{code}

\begin{comment}
\subsection{Values}

The values in this reduced language consist only of algebraic data
types, and functions. A value is either a function, or a constructor
and a list of component values. The type of a value can be deduced
statically -- whether it is a function or an algebraic value, and in
the second case, what are its possible constructors.

\subsection{Expressions}

Expressions in reduced Haskell are defined in Figure \ref{fig:ast}.
A convertor from a subset of Haskell to this reduced language has
been written, and all examples from here onwards use this convertor.

\begin{figure}
\begin{tabular}{lcll}
$E$ & ::= & \T{arg} $m$ (written \T{@}$m$)& $|$ \T{sel} $E \gap C \gap m$ (written $E$.$C_m$) \\
    & $|$ & \T{make} $C \gap E_1\cdots E_n$ & $|$ \T{func} $f$ \\
    & $|$ & \T{apply} $E_0 \gap E_1\cdots E_n$ & $|$ \T{case} $E_0$ \T{of} \{$C_1$ \T{->} $E_1$\T{;}
    $\cdots$ \T{;} $C_n$ \T{->} $E_n$\}
\end{tabular} \\
$f$ is the name of a function, $C$ is the name of a constructor, $m$ is a positive integer
\caption{Abstract Syntax of expressions in reduced Haskell} %
\label{fig:ast}
\end{figure}
\end{comment}

\subsection{Higher Order Functions}

The current checker is not higher order, and does not allow partial
application.

The checker tries to eliminate higher-order functions by
specialization. A mutually recursive group of functions can be
specialized in their $n$th argument if in all recursive calls this
argument is invariant.

Examples of common functions whose applications can be specialized
in this way include \T{map}, \T{filter}, \T{foldr} and \T{foldl}.

When a function can be specialized, the expression passed as the
$n$th argument has all its free variables passed as extra
arguments, and is expanded in the specialized version. All
recursive calls within the new function are then renamed.

\yesexample

\begin{code}
map f xs = case xs of
                []     -> []
                (a:as) -> f a : map f as

adds x n = map (add n) x
\end{code}

\noindent is transformed into:

\begin{code}
map_adds n xs = case xs of
                     []     -> []
                     (a:as) -> add n a : map_adds n as

adds x n = map_adds n x\noexample
\end{code}

Although this firstification approach is not complete by any
means, it appears to be sufficient for a large range of examples.
Alternative methods are available for full firstification, such as
that detailed by Hughes \cite{hughes:type-spec}, or the
defunctionalisation approach by Reynolds
\cite{defunctionalisation}.


\subsection{Internal Representation}

While the concrete syntax allows the introduction of new variable
names, the internal representation does not. All variables are
referred to using a \textit{selector path} from an argument to the
function.

For example, the internal representation of \T{map} is:

\begin{code}
map f xs = case xs of
                []    -> []
                (_:_) -> f (xs\(\D\)hd) : map f (xs\(\D\)tl)
\end{code}

(Note that the infix \D{} operator here is used to compose paths; it
is \textit{not} the Haskell function composition operator.)


% \section{Overview}
% \label{sec:overview}
%
% \todo Delete me!
%
% The checking process has two main ingredients, a constraint language
% for expressing properties on data structures and some mechanisms for
% generating and manipulating constraints to reflect the definition of
% functions. Both are introduced in detail later on, but first a
% sketch overview of the checking process is given. A diagram of the
% process is given in Figure \ref{fig:overview}
%
% \begin{figure}
%
% \begin{center}
% \begin{picture}(340,070)
%
% \put(000,050){\framebox(120,20){\small{Find non-exhaustive patterns}}}
% \put(140,050){\framebox(80,20){\small{Find callers}}}
% \put(140,010){\makebox(80,20){\small{Backward analysis}}}
% \put(140,000){\makebox(80,20){\small{Fixed pointing}}}
% \put(240,000){\framebox(80,30){\small{Report result}}}
% \put(140,000){\framebox(80,30){ }}
%
% \put(120,060){\vector(1,0){20}}
% \put(160,050){\vector(0,-1){20}}
% \put(200,030){\vector(0,01){20}}
% \put(220,015){\vector(1,0){20}}
%
% % \qbezier(75,40)(75,30)(85,30)
% % \put(85,30){\line(1,0){70}}
% % \qbezier(155,30)(165,30)(165,40)
% % \put(165,40){\line(0,1){120}}
% % \qbezier(155,170)(165,170)(165,160)
% % \put(155,170){\line(-1,0){70}}
% % \qbezier(75,160)(75,170)(85,170)
%
% \end{picture}
% \end{center}
%
% \caption{Checker Overview} %
% \label{fig:overview}
% \end{figure}
%
% At each stage, the information passed ``along the arrows'' is a
% predicate, where the atoms are constraints as introduced in
% \S\ref{sec:constraints}. These constraints can either refer to any
% reduced Haskell expression, or in a special case can refer only to
% parameters to functions.
%
% The initial stage of finding all non-exhaustive case expressions is
% done with a basic syntactic check, at a very local level. Initial
% constraints are generated from these expressions.
%
% Finding all callers is relatively straightforward if all constraints
% are only on arguments to functions. The result of this stage are
% constraints on expressions.
%
% Backward analysis and fixed pointing convert back from constraints
% on expressions into constraints on arguments. Backward analysis
% performs the translation, but without regard to recursive function
% calls. Fixed pointing modifies constraints to reflect the recursive
% calls.
%
% This process continues, until the predicate is reduced to either
% True or False. If the end result is True, then the system is free
% from pattern-match errors. If it is False, then the system
% \textit{may} give rise to pattern errors. The checker is
% conservative.
%
% In practice False is always accompanied by a history of derivations,
% ending in False. These derivations allow the user to gain insight
% into a possible cause of failure.

\section{A Constraint Language}
\label{sec:constraints}

In order to implement a checker that can ensure unfailing patterns,
it is useful to have some way of expressing properties of data
values. A constraint is written as $\tup{e,r,c}$ , where $e$ is an
expression, $r$ is a regular expression over selectors and $c$ is a
set of constructors. Such a constraint asserts that any well-defined
application to $e$ of a path of selectors described by $r$ must
reach a constructor in the set $c$.

These constraints are used as atoms in a predicate language with
conjunction and disjunction, so constraints can be about several
expressions and relations between them. The checker does not require
a negation operator. We also use the term constraint to refer to
logical formulae with constraints as atoms.

\yesexample

Consider the function \T{minimum}, defined as:

\begin{code}
minimum xs = case xs of
                  [x]      -> x
                  (a:b:xs) -> minimum (min a b : xs)

min a b = case a < b of
               True  -> a
               False -> b
\end{code}

Now consider the expression \T{minimum $e$}. The constraint that
must hold for this expression to be safe is \cc{$e$}{:}. This says
that the expression $e$ must reduce to an application of \T{:}, i.e.
a non-empty list. In this example the path was $\lambda$ -- the
empty path.\noexample

\yesexample

Consider the expression \T{map minimum $e$}. In this case the
constraint generated is \c{$e$}{tl\K\D hd}{:}. If we apply any
number (possibly zero) of \T{tl}s to $e$, then apply \T{hd}, we
reach a \T{:} construction. Values satisfying this constraint
include \T{[]} and \T{[[1],[2],[3]]}, but not \T{[[1],[]]}. The
value \T{[]} satisfies this constraint because it is impossible to
apply either \T{tl} or \T{hd}, and therefore the constraint does not
assert anything about the possible constructors.

\noexample

Constraints divide up into three parts -- the \textit{subject},
the \textit{path} and the \textit{condition}.

\begin{description}
\item[The subject] in the above two examples was just $e$,
representing any expression -- including a call, a construction or
even a \T{case}.

\item[The path] is a regular expression over selectors.

A regular expression is defined as:\\ \\
\begin{tabular}{ll}
$s+t$ & union of regular expressions $s$ and $t$ \\
$s\D t$ & concatenation of regular expressions $s$ then $t$ \\
$s\K$  & any number (possibly zero) occurrences of $s$ \\
\T{x} & a selector, such as \T{hd} or \T{tl} \\
$\lambda$ & the language is the set containing the empty string \\
$\phi$ & the language is the empty set
\end{tabular}

\item[The condition] is a set of constructors which, due to static type
checking, must all be of the same result type.
\end{description}

The meaning of a constraint is defined by:

\[ \tup{e,r,c} \Leftrightarrow (\forall l \in L(r) \bullet
\text{\textit{defined}}(e,l) \Rightarrow
\text{\textit{constructor}}(e\D{}l) \in c )
\]

\noindent Here $L(r)$ is the language represented by the regular
expression $r$; \textit{defined} returns true if a path selection is
well-defined; and \textit{constructor} gives the constructor used to
create the data. Of course, since $L(r)$ is potentially infinite,
this cannot be checked by enumeration.

If no path selection is well-defined then the constraint is
vacuously true.

\subsection{Simplifying the Constraints}
\label{sec:constraint_simplify}

From the definition of the constraints it is possible to construct a
number of identities which can be used for simplification.

\begin{description}

\item[Path does not exist:] in the constraint \c{[]}{hd}{:} the expression
\T{[]} does not have a \T{hd} path, so this constraint simplifies
to true.

\item[Detecting failure:] the constraint \cc{[]}{:} simplifies to false
because the \T{[]} value is not the constructor \T{:}.

\item[Empty path:] in the constraint $\tup{e,\phi,c}$, the
regular expression is $\phi$, the empty language, so the constraint
is always true.

\item[Exhaustive conditions:] in the constraint \cc{$e$}{:,[]}
the condition lists all the possible constructors, if $e$ reaches
weak head normal form then because of static typing $e$ must be one
of these constructors, therefore this constraint simplifies to true.

\item[Algebraic conditions:] finally a couple of algebraic equivalences:

\[
\begin{array}{rcl}
\tup{e,r_1,c} \wedge \tup{e,r_2,c} & = & \tup{e,(r_1+r_2),c} \\
%incorrect, too conservative
%\tup{e,r,c_1} \vee   \tup{e,r,c_2} & = & \tup{e,r,c_1 \cup c_2} \\
\tup{e,r,c_1} \wedge \tup{e,r,c_2} & = & \tup{e,r,c_1 \cap c_2}
\end{array}
\]
\end{description}


\section{Determining the Constraints}
\label{sec:determining}

This section concerns the derivation of the constraints, and the
operations involved in this task.

\subsection{The Initial Constraints}

In general, a \T{case} expression, where $\vecto{v}$ are the
arguments to a constructor:

\begin{code}
case \(e\) of {C\s{1} \(\vecto{v}\) -> val\s{1}; \ldots; C\s{n} \(\vecto{v}\) -> val\s{n}}
\end{code}

\noindent produces the initial constraint
\cc{$e$}{C\s{1},\ldots,C\s{n}}. If the case alternatives are
exhaustive, then this can be simplified to true. All \T{case}
expressions in the program are found, their initial constraints
are found, and these are joined together with conjunction.

\subsection{Transforming the constraints}

For each constraint in turn, if the subject is \T{x\s{f}} (i.e. the
\T{x} argument to \T{f}), the checker searches for every application
of \T{f}, and gets the expression for the argument \T{x}. On this
expression, it sets the existing constraint. This constraint is then
transformed using a backward analysis (see \S\ref{sec:backward}),
until a constraint on arguments is found.

\yesexample

Consider the constraint \cc{xs\s{minimum}}{:} -- that is
\T{minimum}'s argument \T{xs} must be a non-empty list. If the
program contains the expression:

\begin{code}
f x = minimum (g x)
\end{code}

\noindent then the derived constraint is \cc{(g x\s{f})}{:}.
\noexample

\subsection{Backward Analysis}
\label{sec:backward}

Backward analysis takes a constraint in which the subject is a
compound expression, and derives a combination of constraints over
arguments only. This process is denoted by a function $\varphi$,
which takes a constraint and returns a predicate over constraints.
This function is detailed in Figure~\ref{fig:backward}.

In this figure, $C$ denotes a constructor, $c$ is a set of
constructors, $f$ is a function, $e$ is an expression, $r$ is a
regular expression over selectors and $s$ is a selector.

\begin{figure}

\renewcommand\theequation{sel}
\begin{equation}
    \varphi\tup{e\D{}s,r,c} \rightarrow
    \varphi\tup{e,s\D{}r,c}
\end{equation}

\renewcommand\theequation{con}
\begin{equation}
\frac
    {
        \bigwedge_{i=1}^{\#\vecto{e}}
        \varphi\tup{e_i,\frac{\partial r}{\partial \mathcal{S}(C,i)},c}
        \rightarrow P
    }
    {
        \varphi\tup{C \gap \vecto{e},r,c}
        \rightarrow (\lambda \in L(r) \Rightarrow C \in c) \wedge P
    }
\end{equation}

\renewcommand\theequation{app}
\begin{equation}
    \varphi\tup{f \gap \vecto{e},r,c} \rightarrow
    \varphi\tup{\mathcal{D}(f, \vecto{e}),r,c}
\end{equation}

\renewcommand\theequation{cas}
\begin{equation}
\frac
    {
        \bigwedge_{i=1}^{\#\vecto{e}}
        (
            \varphi\tup{e_0,\lambda,\mathcal{C}(C_i)} \vee
            \varphi\tup{e_i,r,c}
        ) \rightarrow P
    }
    {
        \varphi\tup{\T{case } e_0 \T{ of \{}C_1 \gap \vecto{v} \T{->} e_1\T{;} \cdots
        \T{;} C_n \gap \vecto{v} \T{->} e_n \},r,c} \rightarrow P
    }
\end{equation}

\caption{Specification of backward analysis, $\varphi$} %
\label{fig:backward}
\end{figure}

\begin{description}

\item[The (sel) rule] moves the composition from the expression
to the path.

\item[The (con) rule] deals with an application of a constructor $C$.
If $\lambda$ is in the path language the $C$ must be permitted by
the condition. This depends on the \textit{empty word property}
(ewp) \cite{conway}, which can be calculated structurally on the
regular expression.

For each of the arguments to $C$, a new constraint is obtained from
the derivative of the regular expression with respect to that
argument's selector. This is denoted by $\partial r / \partial
\mathcal{S}(C,i)$, where $\mathcal{S}(C,i)$ gives the selector for
the $i$th argument of the constructor $C$. The differentiation
method is based on that described by Conway \cite{conway}. It can be
used to test for membership in the following way:

\[
\begin{array}{rcl}
\lambda \in L(r) & = & \text{ewp}(r) \\
s\D{}r' \in L(r) & = & r' \in L(\partial r / \partial s)
\end{array}
\]

Two particular cases of note are $\partial \lambda / \partial a =
\phi$ and $\partial \phi / \partial a = \phi$.

\item[The (app) rule] uses the notation $\mathcal{D}(f,\vecto{e})$ to
express the result of substituting each of the arguments in
$\vecto{e}$ into the body of the function $f$. The naive application
of this rule to any function with a recursive call will loop
forever. To combat this, if a function is already in the process of
being evaluated with the same constraint, its result is given as
true, and the recursive arguments are put into a special pile to be
examined later on, see \S\ref{sec:fixed_point} for details.

\item[The (cas) rule] generates a conjunct for each alternative.
The function $\mathcal{C}(C)$ returns the set of all other
constructors with the same result type as $C$, i.e.
$\mathcal{C}(\T{[]}) = \g{:}$. The generated condition says either
the subject of the case analysis has a different constructor (so
this particular alternative is not executed in this circumstance),
or the right hand side of the alternative is safe given the
conditions for this expression.
\end{description}

\subsection{Obtaining a Fixed Point}
\label{sec:fixed_point}

We have noted that if a function is in the process of being
evaluated, and its value is asked for again with the same
constraints, then the call is deferred. After backwards analysis has
been performed on the result of a function, there will be a
constraint in terms of the arguments, along with a set of recursive
calls. If these recursive calls had been analyzed further, then the
checking computation would not have terminated.

\yesexample

\begin{code}
mapHead xs = case xs of
                  []     -> []
                  (x:xs) -> head x : mapHead xs
\end{code}

The function \T{mapHead} is exactly equivalent to \T{map head}.
Running backward analysis over this function, the constraint
generated is \c{xs\s{mapHead}}{hd}{:}, and the only recursive call
noted is \T{mapHead (xs\D{}tl}). The recursive call is written as
\T{xs $\mapsfrom$ xs\D{}tl}, showing how the value of \T{xs}
changes. Observe that the path in the constraint only reaches the
first element in the list, while the desired constraint would reach
them all. In effect \T{mapHead} has been analyzed without
considering any recursive applications.

The fixed point for this function can be derived by repeatedly
replacing \T{xs} with \T{xs\D{}tl} in the subject of the constraint,
and joining these constraints with conjunction.

\setcounter{equation}{0}
\renewcommand\theequation{\arabic{equation}}
\begin{eqnarray}
&& \small{\c{xs}{hd}{:} \wedge \c{xs\D{}tl}{hd}{:} \wedge
   \c{xs\D{}tl\D{}tl}{hd}{:} \wedge \ldots} \\
& \equiv & \small{\c{xs}{hd}{:} \wedge \c{xs}{tl\D{}hd}{:} \wedge
   \c{xs}{tl\D{}tl\D{}hd}{:} \wedge \ldots} \\
& \equiv & \small{\c{xs}{hd+tl\D{}hd+tl\D{}tl\D{}hd+$\ldots$}{:}} \\
& \equiv & \small{\c{xs}{($\lambda$+tl+tl\D{}tl+$\ldots$)\D{}hd}{:}} \\
& \equiv & \small{\c{xs}{tl\K\D{}hd}{:}}
\end{eqnarray}

The justification is as follows. First use the backwards analysis
rule given in Figure \ref{fig:backward} to transform between (1) and
(2) -- selectors move from the subject to the path. To obtain (3)
the first algebraic condition given in
\S\ref{sec:constraint_simplify} is used. The factorisation of the
\T{hd} element of the regular expression is applied. Finally this
can be rewritten using the regular expression \K operator as the
result. \noexample

More generally, given any constraint of the form \tup{\T{x},r,c} and
a recursive call of the form \T{x $\mapsfrom$ x.$p$}, the fixed
point is \tup{$x$,p\K\D{}r,c}. A special case is where $p$ is
$\lambda$, in which case $p$\K\D{}$r = r$.

\yesexample

Consider the function \T{reverse} written using an accumulator:

\begin{code}
reverse x = reverse2 x []

reverse2 x a = case x of
                    []     -> a
                    (y:ys) -> reverse2 ys (y:a)
\end{code}

Argument \T{xs} follows the pattern \T{x $\mapsfrom$ x.tl}, but we
also have the recursive call \T{a $\mapsfrom$ (x\D{}hd:a)}. If the
program being analyzed contained an instance of \T{map head (reverse
x)}, the part of the condition that applies to \T{a} before the
fixed pointing of \T{a} is \c{a}{tl\K\D{}hd}{:}.

In this case a second rule for obtaining a fixed point can be used.
For recursive calls of the form \T{a $\mapsfrom$ C x$_1$ $\cdots$
x$_n$ a}, where $s$ is the selector corresponding to the position of
$a$, the rule is:

\[\bigwedge_{r'\in r^{\#}}\left(
    \left(\lambda\in L(r') \Rightarrow C \in c \right) \wedge
    \tup{a,r',c} \wedge
    \bigwedge_{i=1}^{n}\tup{x_i,\frac{\partial r'}{\partial
    \mathcal{S}(C,i)},c}
    \right)
\]

Where:

\[
r^{\#} = \{r^0,r^1,\ldots,r^{\infty}\}
\hspace{1cm} r^0 = r \hspace{1cm}
r^{(n+1)} = \frac{\partial r^n}{\partial s}
\]

It can be shown that $r^{\#}$ is always a finite set
\cite{lawson:finite_automata}. This expression is derived from the
(con) rule \S\ref{sec:backward}, applied until it reaches a fixed
point.

In the \T{reverse} example, $r^{\#}$ is $\{\T{tl\K\D{}hd}\}$, since
$\partial \T{tl\K\D{}hd} / \partial \T{tl} = \T{tl\K\D{}hd}$. Also
$\lambda \notin L(\T{tl\K\D{}hd})$, so the result is:

\begin{eqnarray*}
&& \c{a}{tl\K\D{}hd}{:} \wedge
   \c{x\D{}hd}{$\frac{\partial \T{tl\K\D{}hd}}{\partial
   \T{hd}}$}{:} \\
& \equiv & \c{a}{tl\K\D{}hd}{:} \wedge \cc{x\D{}hd}{:} \\
& \equiv & \c{a}{tl\K\D{}hd}{:} \wedge \c{x}{hd}{:}
\end{eqnarray*}

Next applying the fixed pointing due to \T{x}, gives a final
condition, as expected:

\noindent \c{a}{tl\K\D{}hd}{:} $\wedge$ \c{x}{tl\K\D{}hd}{:}
\noexample

While the two rules given do cover a wide range of examples, they
are not complete. Additional rules exist for other forms of
recursion but not all recursive functions can be handled using the
current scheme.

\yesexample

\begin{code}
interleave x y = case x of
                      []    -> y
                      (a:b) -> a : interleave y b
\end{code}

Here the recursive call is \T{y $\mapsfrom$ x\D{}tl}, which does not
have a rule defined for it. In such cases the checker conservatively
outputs \T{False}, and also gives a warning message to the user. The
checker always terminates.

The fixed point rules classify exactly which forms of recursion can
be accepted by the checker. Defining more fixed point rules which
can capture an increasingly large number of patterns is a matter for
future work.

\section{A Worked Example}
\label{sec:worked_example}

Recall the \T{risers} example in \S\ref{sec:introduction}. The first
step of the checker is to transform this into reduced Haskell.

\begin{code}
risers xs =
    case xs of
         []        -> []
         [x]       -> [[x]]
         (x:y:etc) -> risers2 (x <= y) x (risers (y:etc))

risers2 b x y = case y of
                     (s:ss) -> case b of
                                    True  -> (x:s) : ss
                                    False -> [x] : (s:ss)
\end{code}

The auxiliary \T{risers2} is necessary because reduced Haskell has
no \T{where} clause. The checker proceeds as follows:

\paragraph{Step 1, Find all incomplete case statements.} The checker
finds one, in the body of \T{risers2}, the argument \T{y} must be a
non-empty list. The constraint is \cc{y\s{risers2}}{:}.

\paragraph{Step 2, Propagate.} The auxiliary \T{risers2} is
applied by \T{risers} with \T{risers (y:etc)} as the argument \T{s}.
This gives \cc{(risers (y:etc))}{:}. When rewritten in terms of
arguments and paths of selectors, this gives the constraint
\cc{(risers (xs\s{risers}\D tl\D hd :$\gap{}$xs\s{risers}\D tl\D
tl))}{:}.

\paragraph{Step 3, Backward analysis.} The constraint is transformed
using the backward analysis rules. The first rule invoked is (app),
which says that the body of \T{risers} must evaluate to a non-empty
list, in effect an inline version of the constraint. Backward
analysis is then performed over the case statement, the
constructors, and finally \T{risers2}. The conclusion is that
provided \T{xs\s{risers}} is a \T{:}, the result will be. The
constraint is \cc{(xs\s{risers}\D tl\D hd :$\gap{}$xs\s{risers}\D
tl\D tl)}{:}, which is true.

In this example, there is no need to perform any fixed pointing.

\section{Some Small Examples and a Case Study}
\label{sec:examples}

In the following examples, each line represents one propagation step
in the checker. The final constraint is given on the last line.

\begin{code}
head x = case x of
              (y:ys) -> y
main x = head x
> \cc{x\s{head}}{:}
> \cc{x\s{main}}{:}
\end{code}

\noindent This example requires only initial constraint generation,
and a simple propagation. \noexample


%
% \yesexample
%
% \begin{code}
% list x = case x of
%              (a:as) -> head x : tail x
%              _ -> []
% main x = list x
% > head@1\g{:} \(\wedge\) tail@1\g{:}
% > True
% \end{code}
%
% \noindent In this example \T{list} is about as useful as \T{id}, but
% it does show some nice features of the system. Even though
% internally \T{list} uses the non-exhaustive functions \T{head} and
% \T{tail}, the checker is able to tell that their arguments are not
% the empty list. This example shows the use of non-exhaustive
% components in a program which is proved to be free from
% pattern-match errors. \noexample

\yesexample

\begin{code}
main x = map head x
> \cc{x\s{head}}{:}
> \c{x\s{map_head}}{tl\K\D{}hd}{:}
> \c{x\s{main}}{tl\K\D{}hd}{:}
\end{code}

\noindent This example shows specialization generating a new
function \verb"map_head", fixed pointing being applied to \T{map},
and the constraints being propagated through the system. \noexample


\yesexample

\begin{code}
main x = map head (reverse x)
> \cc{x\s{head}}{:}
> \c{x\s{map_head}}{tl\K\D{}hd}{:}
> \c{x\s{main}}{tl\K}{:} \(\vee\) \c{x\s{main}}{tl\K\D{}hd}{:}
\end{code}

\noindent This result may at first seem surprising. The first
disjunct of the constraint says that applying \T{tl} any number of
times to \T{x\s{main}} the result must always be a \T{:}, in other
words \T{x} must be infinite. This guarantees case safety because
\T{reverse} is tail strict, so if its argument is an infinite list,
no result will ever be produced, and a case error will not occur.
The second disjunct says, less surprisingly, that every item in
\T{x} must be a non-empty list. \noexample

\yesexample

\begin{code}
main xs ys = case null xs || null ys of
                  True -> 0
                  False -> head xs + head ys
> \cc{x\s{head}}{:}
> \cc{(null xs\s{main} || null ys\s{main})}{True} \(\vee\)
  (\cc{xs\s{main}}{:} \(\wedge\) \cc{ys\s{main}}{:})
> \cc{xs\s{main}}{[]} \(\vee\) \cc{ys\s{main}}{[]} \(\vee\) (\cc{xs\s{main}}{:} \(\wedge\) \cc{ys\s{main}}{:})
> True
\end{code}

This example shows the use of a more complex condition to guard a
potentially unsafe application of \T{head}. The backward analysis
applied to \T{null} and \T{||} gives precise requirements, which
when expanded results in a tautology, showing that no pattern match
error can occur.\noexample

\yesexample

\begin{code}
main x = tails x
tails x = foldr tails2 [[]] x
tails2 x y = (x:head y) : y
> \cc{x\s{head}}{:}
> \cc{y\s{tails2}}{:}
> \cc{n1\s{foldr_tails2}}{:} \(\vee\) \c{n2\s{foldr_tails2}}{tl\K\D{}tl}{:}
> True
\end{code}

\noindent This final example uses a fold to calculate the \T{tails}
function. As the auxiliary \T{tails2} makes use of \T{head} the
program is not obviously free from pattern-match errors. The first
two lines of the output are simply moving the constraint around. The
third line is the interesting one. In this line the checker gives
two alternative conditions for the case safety of \T{foldr tails2}
-- either its first argument is a \T{:}, or its second argument is
empty or infinite. The way the requirement for empty or infinite
length is encoded is by the path \T{tl\K\D{}tl}. If the list is
\T{[]}, then there are no tails to match the path. If however, there
is one tail, then that tail, and all successive tails must be \T{:}.
So either \T{foldr} does not call its function argument because it
immediately takes the \T{[]} case, or \T{foldr} recurses infinitely,
and therefore the function is never called. Either way, because
\T{foldr}'s second argument is a \T{:}, and because \T{tails2}
always returns a \T{:}, the first part of the condition can be
satisfied. \noexample

\subsection{The Clausify Program}

Our goal is to check standard Haskell programs, and to provide
useful feedback to the user. To test the checker against these
objectives we have used several Haskell programs, all written some
time ago for other purposes. The analysis of one program is
discussed below.

The Clausify program has been around for a very long time, since at
least 1990. It has made its way into the \T{nofib} benchmark suite
\cite{nofib}, and was the focus of several papers on heap profiling
\cite{clausify}. It parses logical propositions and puts them in
clausal form. We ignore the parser and jump straight to the
transformation of propositions. The data structure for a formula is:

\begin{code}
data F = Sym \{char :: Char\} | Not \{n :: F\}
       | Dis \{d1, d2 :: F\} | Con \{c1, c2 :: F\}
       | Imp \{i1, i2 :: F\} | Eqv \{e1, e2 :: F\}
\end{code}

\noindent and the main pipeline is:

\begin{code}
unicl . split . disin . negin . elim
\end{code}

Each of these stages takes a proposition and returns an equivalent
version -- for example the \T{elim} stage replaces implications with
disjunctions and negation. Each stage eliminates certain forms of
proposition, so that future stages do not have to consider them.
Despite most of the stages being designed to deal with a restricted
class of propositions, the only function which contains a
non-exhaustive pattern match is in the definition of \T{clause} (a
helper function for \T{unicl}).

\begin{code}
clause p = clause' p ([] , [])
    where
    clause' (Dis p q)       x   = clause' p (clause' q x)
    clause' (Sym s)       (c,a) = (insert s c , a)
    clause' (Not (Sym s)) (c,a) = (c , insert s a)
\end{code}

After encountering the non-exhaustive pattern match, the checker
generates the following constraints:

\begin{code}
> \c{p\s{clause'}}{(d1+d2)\K}{Dis,Sym,Not} \(\wedge\) \c{p\s{clause'}}{(d1+d2)\K\D{}n}{Sym}
> \c{p\s{clause'}}{(d1+d2)\K}{Dis,Sym,Not} \(\wedge\) \c{p\s{clause}}{(d1+d2)\K\D{}n}{Sym}
> \c{p\s{unicl'}}{(d1+d2)\K}{Dis,Sym,Not} \(\wedge\) \c{p\s{unicl'}}{(d1+d2)\K\D{}n}{Sym}
> \c{x\s{foldr_unicl}}{tl\K\D{}hd\D(d1+d2)\K}{Dis,Sym,Not} \(\wedge\)
  \c{x\s{foldr_unicl}}{tl\K\D{}hd\D(d1+d2)\K\D{}n}{Sym}
> \c{x\s{unicl}}{tl\K\D{}hd\D(d1+d2)\K}{Dis,Sym,Not} \(\wedge\)
  \c{x\s{unicl}}{tl\K\D{}hd\D(d1+d2)\K\D{}n}{Sym}
\end{code}

These constraints give accurate and precise requirements for a case
error not to occur at each stage. However, when the condition is
propagated back over the \T{split} function, the result becomes less
pleasing. None of our fixed pointing schemes handle the original
recursive definition of \T{split}:

\begin{code}
split p = split' p []
    where
    split' (Con p q) a = split' p (split' q a)
    split' p a = p : a
\end{code}

\noindent can be transformed manually by the removal of the
accumulator:

\begin{code}
split (Con p q) = split p ++ split q
split p = [p]
\end{code}

This second version is accepted by the checker, which generates the
constraint:

\begin{code}
> \c{p\s{split}}{(c1+c2)\K}{Con,Dis,Sym,Not} \(\wedge\)
  \c{p\s{split}}{(c1+c2)\K\D(d1+d2)\D(d1+d2)\K}{Dis,Sym,Not} \(\wedge\)
  \c{p\s{split}}{(c1+c2)\K\D(d1+d2)\K\D{}n}{Sym}
\end{code}

This constraint can be read as follows: the outer structure of a
propositional argument to \T{split} is any number of nested \T{Con}
constructors; the next level is any number of nested \T{Dis}
constructors; at the innermost level there must be either a \T{Sym},
or a \T{Not} containing a \T{Sym}. That is, propositions are in
\textit{conjunctive normal form}.

The one surprising part of this constraint is the
\T{(d1+d2)\D(d1+d2)\K} part of the path in the 2nd conjunct. We
might rather expect something similar to
\T{(c1+c2)\K\D(d1+d2)\K\{Dis,Sym,Not\}}, but consider what this
means. Take as an example the value \T{(Con (Sym 'x') (Sym 'y'))}.
This value meets all 3 conjunctions generated by the tool, but does
not meet this new constraint: the path has the empty word property,
so the root of the value can no longer be a \T{Con} constructor.

The next function encountered is \T{disin} which shifts
disjunction inside conjunction. The version in the nofib benchmark
has the following equation in its definition:

\begin{code}
disin (Dis p q) = if conjunct dp || conjunct dq
                  then disin (Dis dp dq)
                  else (Dis dp dq)
    where
    dp = disin p
    dq = disin q
\end{code}

Unfortunately, when expanded out this gives the call

\begin{code}
disin (Dis (disin p) (disin q))
\end{code}

\noindent which does not have a fixed point under the present
scheme. Refactoring is required to enable this stage to succeed.
Fortunately, in \cite{clausify} a new version of \T{disin} is given,
which is vastly more efficient than this one, and (as a happy side
effect) is also accepted by the checker.

At this point the story comes to an end. Although a constraint is
calculated for the new \T{disin}, this constraint is approximately
15 printed pages long! Initial exploration suggests at least one
reason for such a large constraint: there are missed opportunities
to simplify paths. We are confident that with further work the
Clausify example can be completed.

\section{Related Work}
\label{sec:related_work}

Viewed as a \textbf{proof tool} this work can be seen as following
Turner's goal to define a Haskell-like language which is total
\cite{tfp:total}. Turner disallows incomplete pattern matches,
saying this will ``force you to pay attention to exactly those
corner cases which are likely to cause trouble''. Our checker may
allow this restriction to be lifted, yet still retain a total
programming language.

Viewed as a basic \textbf{pattern match checker}, the work on
compiling warnings about incomplete and overlapping patterns is
quite relevant \cite{ghc,pattern_match}. As noted in the
introduction, these checks are only local.

Viewed as a \textbf{mistake detector} this tool has a similar
purpose to the classic C Lint tool \cite{lint}, or Dialyzer
\cite{dialyzer} -- a static checker for Erlang. The aim is to have a
static checker that works on unmodified code, with no additional
annotations. However, a key difference is that in Dialyzer all
warnings indicate a genuine problem that needs to be fixed. Because
Erlang is a dynamically typed language, a large proportion of
Dialyzer's warnings relate to mistakes a type checker would have
detected.

Viewed as a \textbf{soft type system} the checker can be compared to
the tree automata work done on XML and XSL \cite{static_xslt}, which
can be seen as an algebraic data type and a functional language.
Another soft typing system with similarities is by Aiken
\cite{type:dynamic}, on the functional language FL. This system
tries to assign a type to each function using a set of constructors,
for example \T{head} is given just \T{Cons} and not \T{Nil}.

\section{Conclusions and Further Work}
\label{sec:conclusions}

A static checker for potential pattern-match errors in Haskell has
been specified and implemented. This checker is capable of
determining preconditions under which a program with non-exhaustive
patterns executes without failing due to a pattern-match error. A
range of small examples has been investigated successfully. Where
programs cannot be checked initially, refactoring can increase the
checker's success rate. Work in progress includes:

\begin{itemize}

\item The checker currently relies on specialization to remove higher order
functions.

\item The checker is fully polymorphic but it does not currently
handle Haskell's type classes; we hope these can be transformed away
without vast complication \cite{class_remove}.

\item Another challenge is to translate from full Haskell into the
reduced language. This work has been started: we have a converter
for a useful subset.

\item The checker should offer fuller traces that can
be manually verified. Currently the predicate at each stage is
given, without any record of how it was obtained, or what effect
fixed pointing had. Although a more detailed trace would not help an
end user, it would help strengthen the understanding of the
algorithms.

\item The central algorithms of the checker can be refined. In particular
a better fixed pointing scheme is being developed. A complete
analysis of which programs can be verified would be useful.

\item A correctness proof is needed to prove that the checker is
sound. This will require a semantics for the reduced Haskell-like
language.

\end{itemize}

With these improvements we hope to check larger Haskell programs,
and to give useful feedback to the programmer.

\section*{Acknowledgement}

The first author is a PhD student supported by a studentship from
the Engineering and Physical Sciences Research Council of the UK.

% \nocite{*}
\bibliographystyle{abbrv}
\bibliography{case_check}

\end{document}
