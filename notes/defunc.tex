\documentclass{article}

\title{CATCH: Defunctionalization}
\author{Neil Mitchell}

\usepackage{alltt}
\usepackage{fullpage}

\newcommand{\T}[1]{\texttt{#1}}

\newenvironment{code}{\begin{alltt}\small}{\end{alltt}}
\newenvironment{codepage}
    {\begin{minipage}[h]{\textwidth}\begin{code}}
    {\end{code}\end{minipage}}

\newcommand\arr{$\rightarrow$}
\newcommand\lam{$\lambda$}


\begin{document}

\maketitle

\section{Purpose}

This section is meant to show how defunctionalization can transform a program
with higher order functions into a first order one.

\section{Remove Anonymous Lambda}

These should be removed, relatively easy.

\section{Calculating Arity}

Before Partial Applications can be removed, it must first be determined what
are partial and which are saturated. This requires the arity of functions to be
calculated.

In the following example:

\begin{code}
> even = (.) not odd
> (.) f g x = f (g x)
\end{code}

\T{(.)} takes 3 arguments, therefore the arity of \T{(.)} is 3. \T{even} has
\T{(.)} at its root, with 2 arguments, therefore the value is the arity of
\T{(.)} minus 2 = 1.

\section{Remove Partial Application}

If a function has less arguments than it needs, add a lambda.

In the above example, \T{not} and \T{odd} are both calculated to have arity 1:

\begin{code}
> even = \lam{}x -> (.) (\lam{}y -> not y) (\lam{}z -> odd z) x
\end{code}

This can be transformed to:

\begin{code}
> even x = (.) (\lam{}y -> not y) (\lam{}z -> odd z) x
\end{code}

\section{Apply Reynolds method}

A solution to this is to create partial application data structures everywhere:

\begin{code}
> data Func1 = Not1
>            | Odd1
>
> apply1 :: Func1 -> a -> b
> apply1 Not1 x = not x
> apply1 Odd1 x = odd x
\end{code}

Now these can be transformed in, replacing all evaluations of a higher order
function with an apply:

\begin{code}
> even x = (.) Not1 Odd1 x
> (.) f g x = apply1 f (apply1 g x)
\end{code}

Now there are no higher order functions left.

\end{document}
